\begin{table}[ht]
\centering
\begin{tabular}{cccccccc}
  \hline
P & mgcv & LASSO & COSSO & Adaptive COSSO & BHAM & SB-GAM & spikeSlabGAM \\ 
  \hline
  4 & 0.94 (0.01) & 0.83 (0.01) & 0.90 (0.01) & 0.90 (0.01) & 0.92 (0.01) & 0.92 (0.01) & 0.90 (0.00) \\ 
   10 & 0.92 (0.03) & 0.83 (0.00) & 0.86 (0.04) & 0.86 (0.03) & 0.92 (0.01) & 0.92 (0.01) & 0.90 (0.00) \\ 
   50 & 0.76 (0.03) & 0.83 (0.01) & 0.83 (0.02) & 0.84 (0.02) & 0.90 (0.01) & 0.92 (0.01) & 0.89 (0.01) \\ 
  100 & NaN (NA) & 0.83 (0.01) & 0.83 (0.02) & 0.81 (0.09) & 0.90 (0.01) & 0.92 (0.01) & 0.88 (0.01) \\ 
  200 & NaN (NA) & 0.83 (0.01) & 0.81 (0.06) & 0.82 (0.05) & 0.88 (0.02) & 0.92 (0.01) & 0.87 (0.02) \\ 
   \hline
\end{tabular}
\caption{The average and standard deviation of the out-of-sample area under the curve measures
    for binomial outcomes over 50 iterations. The models of comparison include the proposed Bayesian
    hierarchical additive model (BHAM) fitted with Iterative Weighted Least Square (BHAM-IWLS) and
    Coordinate Descent (BHAM-CD) algorithms, component selection and smoothing operator (COSSO),
    adaptive COSSO, mgcv and sparse Bayesian generalized additive model (SB-GAM). mgcv doesn't provide
    estimation whe number of parameters exceeds sample size i.e. p = 100, 200.} 
\label{tab:bin_auc}
\end{table}
