% Options for packages loaded elsewhere
\PassOptionsToPackage{unicode}{hyperref}
\PassOptionsToPackage{hyphens}{url}
%
\documentclass[
]{article}
\title{Spike-and-Slab LASSO Generalized Additive Models and Scalable
Algorithms for High-Dimensional Data Analysis}
\usepackage{etoolbox}
\makeatletter
\providecommand{\subtitle}[1]{% add subtitle to \maketitle
  \apptocmd{\@title}{\par {\large #1 \par}}{}{}
}
\makeatother
\subtitle{Supporting Information}
\author{Boyi Guo, Byron C. Jaeger, AKM Fazlur Rahman, D. Leann Long,
Nengjun Yi}
\date{}

\usepackage{amsmath,amssymb}
\usepackage{lmodern}
\usepackage{iftex}
\ifPDFTeX
  \usepackage[T1]{fontenc}
  \usepackage[utf8]{inputenc}
  \usepackage{textcomp} % provide euro and other symbols
\else % if luatex or xetex
  \usepackage{unicode-math}
  \defaultfontfeatures{Scale=MatchLowercase}
  \defaultfontfeatures[\rmfamily]{Ligatures=TeX,Scale=1}
\fi
% Use upquote if available, for straight quotes in verbatim environments
\IfFileExists{upquote.sty}{\usepackage{upquote}}{}
\IfFileExists{microtype.sty}{% use microtype if available
  \usepackage[]{microtype}
  \UseMicrotypeSet[protrusion]{basicmath} % disable protrusion for tt fonts
}{}
\makeatletter
\@ifundefined{KOMAClassName}{% if non-KOMA class
  \IfFileExists{parskip.sty}{%
    \usepackage{parskip}
  }{% else
    \setlength{\parindent}{0pt}
    \setlength{\parskip}{6pt plus 2pt minus 1pt}}
}{% if KOMA class
  \KOMAoptions{parskip=half}}
\makeatother
\usepackage{xcolor}
\IfFileExists{xurl.sty}{\usepackage{xurl}}{} % add URL line breaks if available
\IfFileExists{bookmark.sty}{\usepackage{bookmark}}{\usepackage{hyperref}}
\hypersetup{
  pdftitle={Spike-and-Slab LASSO Generalized Additive Models and Scalable Algorithms for High-Dimensional Data Analysis},
  pdfauthor={Boyi Guo, Byron C. Jaeger, AKM Fazlur Rahman, D. Leann Long, Nengjun Yi},
  hidelinks,
  pdfcreator={LaTeX via pandoc}}
\urlstyle{same} % disable monospaced font for URLs
\usepackage[margin=1in]{geometry}
\usepackage{graphicx}
\makeatletter
\def\maxwidth{\ifdim\Gin@nat@width>\linewidth\linewidth\else\Gin@nat@width\fi}
\def\maxheight{\ifdim\Gin@nat@height>\textheight\textheight\else\Gin@nat@height\fi}
\makeatother
% Scale images if necessary, so that they will not overflow the page
% margins by default, and it is still possible to overwrite the defaults
% using explicit options in \includegraphics[width, height, ...]{}
\setkeys{Gin}{width=\maxwidth,height=\maxheight,keepaspectratio}
% Set default figure placement to htbp
\makeatletter
\def\fps@figure{htbp}
\makeatother
\setlength{\emergencystretch}{3em} % prevent overfull lines
\providecommand{\tightlist}{%
  \setlength{\itemsep}{0pt}\setlength{\parskip}{0pt}}
\setcounter{secnumdepth}{-\maxdimen} % remove section numbering
\newlength{\cslhangindent}
\setlength{\cslhangindent}{1.5em}
\newlength{\csllabelwidth}
\setlength{\csllabelwidth}{3em}
\newlength{\cslentryspacingunit} % times entry-spacing
\setlength{\cslentryspacingunit}{\parskip}
\newenvironment{CSLReferences}[2] % #1 hanging-ident, #2 entry spacing
 {% don't indent paragraphs
  \setlength{\parindent}{0pt}
  % turn on hanging indent if param 1 is 1
  \ifodd #1
  \let\oldpar\par
  \def\par{\hangindent=\cslhangindent\oldpar}
  \fi
  % set entry spacing
  \setlength{\parskip}{#2\cslentryspacingunit}
 }%
 {}
\usepackage{calc}
\newcommand{\CSLBlock}[1]{#1\hfill\break}
\newcommand{\CSLLeftMargin}[1]{\parbox[t]{\csllabelwidth}{#1}}
\newcommand{\CSLRightInline}[1]{\parbox[t]{\linewidth - \csllabelwidth}{#1}\break}
\newcommand{\CSLIndent}[1]{\hspace{\cslhangindent}#1}
\ifLuaTeX
  \usepackage{selnolig}  % disable illegal ligatures
\fi

\begin{document}
\maketitle

\newcommand{\tp}{*}
\newcommand{\pr}{\text{Pr}}
\newcommand{\bs}[1]{\boldsymbol{#1}}
\newcommand{\simiid}{\overset{\text{iid}}{\sim}}

\subsection{Supplementary Information 1: Marginal Distribution of $\gamma_j^*$}

Given that
\(\gamma_{j}^*| \gamma_{j}, \theta_j \sim Bin(1, \gamma_{j}\theta_j)\)
where \(\gamma_{j} | \theta_j \sim Bin(1, \theta_j)\), we can derive the
marginal distribution of \(\gamma_{j}^*\) with the following
manipulation. \begin{align*}
& \text{Pr}(\gamma_j^*= 1 | \theta_j)  = \text{Pr}(\gamma_j^*= 1, \gamma_j = 1 | \theta_j) + \text{Pr}(\gamma_j^*= 1 , \gamma_j = 0| \theta_j)\\
= & \text{Pr}(\gamma_j^*= 1, \gamma_j = 1 | \theta_j) + 0 \qquad \text{[hierarchical structure between }\gamma^*\text{ and }\gamma \text{.]}\\
= & \text{Pr}(\gamma_j^*= 1| \gamma_j = 1,  \theta_j)\text{Pr}(\gamma_j = 1| \theta_j)\\
= & \theta_j^2\\
& \text{Pr}(\gamma_j^*= 0 | \theta_j)  = \text{Pr}(\gamma_j^*= 0, \gamma_j = 1 | \theta_j) + \text{Pr}(\gamma_j^*= 0 , \gamma_j = 0| \theta_j)\\
= & \text{Pr}(\gamma_j^*= 0, \gamma_j = 1 | \theta_j) + \text{Pr}(\gamma_j^*= 0, \gamma_j = 0 | \theta_j)\\
= & \text{Pr}(\gamma_j^*= 0| \gamma_j = 1,  \theta_j)\text{Pr}(\gamma_j = 1| \theta_j) + \text{Pr}(\gamma_j^*= 0| \gamma_j = 0,  \theta_j)\text{Pr}(\gamma_j = 0| \theta_j)\\
= & (1-\theta_j)\theta_j + 1(1-\theta_j) = 1-\theta_j^2
\end{align*}

\clearpage

\subsection{Supplementary Information 2: EM-Iterative Weighted Least Square Algorithm for BHAM Model Fitting}

Similar to the EM-CD algorithm, the EM-Iterative Weighted Least Square
(EM-IWLS) algorithm is an EM-based algorithm where the iterative
weighted least squares algorithm is used to find the estimate of
\(\boldsymbol{\beta}, \phi\) that maximizes \(E(Q_1)\). The iterative
weighted least squares algorithm was originally proposed to fit the
classical generalized linear models, and generalized to fit some
Bayesian hierarchical models.(Gelman et al. 2013) Yi and Ma (Yi and Ma
2012) extended the algorithm to fit Bayesian hierarchical models for
high-dimensional data analysis. Specifically, the authors formulated
Student's t-distribution and double exponential distribution as
hierarchical normal distributions such that generalized linear models
with shrinkage priors can be easily fitted. In this work, we further
extend the EM-IWLS paradigm to fit the proposed BHAM method. Compare to
the EM-CD algorithm, EM-IWLS estimates the variance-covariance matrix of
the coefficients, providing an opportunity to derive the uncertainty
quantification of smoothing functions. We will defer this discussion to
future work, due to the delicacy of the topic. However, the EM-IWLS is
not as computationally efficient as EM-CD, particularly in
high-dimensional settings.

A double exponential prior, \(\beta|S \sim DE(0, S)\) can be formulated
as a hierarchical normal prior with unknown variance \(\tau^2\)
integrated out: \begin{align*}
  \beta|\tau^2 &\sim N(0, \tau^2)\\
  \tau^2|S & \sim Gamma(1, 1/(2S^2)), 
\end{align*} For the mixture double exponential priors, we can define
the scale parameter \(S = (1-\gamma)s_0 + \gamma s_1\). The change in
the prior formulation in turn leads to the change in the log posterior
density function, as \(Q_1\) needs to account for the hyperprior of
\(\tau^2\): \begin{equation}\label{eq:Q1_IWLS}
Q_1(\boldsymbol{\beta}, \phi) = \log f(\textbf{y}|\boldsymbol{\beta}, \phi) + \sum\limits_{j=1}^p\left[\log f(\beta_j|{\tau}^2_j) + \log f({\tau}^2_j| S_j)+\sum\limits_{k=1}^{K_j} \{\log f(\beta^{*}_{jk}|{\tau^{*}}^2_{jk})+\log f({\tau^*}^2_{jk}| S^*_j)\}\right].
\end{equation} Since \(\boldsymbol{\tau}^2\) are not of our primary
interest, we treat them as the ``missing'' data in addition to the
latent indicators \(\boldsymbol{\gamma}\), and hence construct the
expectation
\(E_{\boldsymbol{\gamma}, \boldsymbol{\tau}^2|\Theta^{(t-1)}}(Q_1)\) in
the E-step. To note, unlike the same latent indicator \(\gamma^*_j\)
which is shared by the coefficients of the nonlinear terms
\(\beta^*_{jk}\) for \(k = 1, \dots, K_j\) , \(\tau^2_{jk}\) is
coefficient specific for \(\beta^*_{jk}\).
\(E({S_j}^{-1}|\beta_j, s_0, s_1), E({S^*}^{-1}_j|\boldsymbol{\beta}_j^*, s_0, s_1), E({\tau}^2_{j}|S_j, \beta_j) \text{ and } E({\tau^*}^2_{jk}|S_j^*, \beta^*_{jk})\)
needs to be calculated to formulate \(E(Q_1)\). As neither
\(E({S_j}^{-1}|\beta_j, s_0, s_1)\) nor
\(E({S^*}^{-1}_j|\boldsymbol{\beta}_j^*, s_0, s_1)\) depends on
\(\tau^2\)s, they can be derived following the same derivation in the
EM-CD algorithm. On the other hand, \(\tau^{2}\), following gamma
distributions, is a conjugate prior for the normal variance, and the
conditional posterior density of \(\tau^{-2}\) is an inverse Gaussian
distribution. \(E({\tau}^{-2}_{j})\) and \(E({\tau^*}^{-2}_{jk})\) are
calculated using the closed form equation \begin{align*}
 E({\tau}^{-2}_{j}|S_j, \beta_j) ={S_j}^{-1}/|\beta_j| \qquad E({\tau^*}^{-2}_{jk}|S_j^*, \beta^*_{jk})={S_j^*}^{-1}/|\beta^*_{jk}|,
\end{align*} where \(S_j\) and \(S_j^*\) are replaced by the expectation
and \(\beta\)s are replaced with \(\beta^{(t-1)}\). With simplification
(up to constant additive terms), we have
\begin{equation}\label{eq:EQ1_IWLS}
E(Q_1) = \log f(\textbf{y}|\boldsymbol{\beta}, \phi) - \sum\limits_{j=1}^p\left[ {2E({\tau_j}^{-2})}{\beta_j}^2 +\sum\limits_{k=1}^{K_j} {2E({\tau_{jk}^*}^{-2})}{\beta_{jk}^*}^2\right].
\end{equation} \(2E({\tau}^{-2})\beta^2\) can be seen as the kernel of a
normal density with mean 0 and variance \(E(\tau^{2})\), and we can
formulate the coefficients \(\boldsymbol{\beta}\) as a multivariate
normal distribution with means \(\boldsymbol{0}\) and variance
covariance matrix \(\boldsymbol{\Sigma}_{\tau^2}\), where
\(\boldsymbol{\Sigma}_{\tau^2}\) is a diagonal matrix with
\(E(\tau^2)\)s on the diagonal, \[
\boldsymbol{\beta }\sim \text{MVN}(0, \boldsymbol{\Sigma}_{\tau^2}).
\]

Meanwhile, following the classical IWLS, we can approximate the
generalized model likelihood at each iteration with a weighted normal
likelihood: \[
f(\textbf{y}|\boldsymbol{\beta}, \phi) \approx \text{MVN}(\textbf{z}|\boldsymbol{X} \boldsymbol{\beta}, \phi\boldsymbol{\Sigma })
\] where the `normal response' \(z_i\) and `weight' \(w_i\) are called
the pseudo-response and pseudo-weight respectively. The pseudo-response
and the pseudo-weight are calculated by \[
\begin{aligned}
z_i &= \hat\eta_i - \frac{L^{'}(y_i|\hat\eta_i)}{L^{''}(y_i|\hat\eta_i)}& w_i &= - L^{''}(y_i|\hat\eta_i),
\end{aligned}
\] where \(\hat\eta_i = (\boldsymbol{X} {\hat{\boldsymbol{\beta}}})_i\),
\(L^{'}(y_i|\hat\eta_i, \hat \phi)\) and
\(L^{''}(y_i|\hat\eta_i, \hat \phi)\) are the first and second
derivative of the log density,
\(\log f(\textbf{y}_i|\boldsymbol{\beta}, \phi)\) with respect to
\(\eta_i\).

With
\(\boldsymbol{z}\sim \text{MVN}(\boldsymbol{X} \boldsymbol{\beta}, \phi \boldsymbol{\Sigma})\)
and
\(\boldsymbol{\beta }\sim \text{MVN}(0, \phi \boldsymbol{\Sigma}_{\tau^2})\),
we can augment the two multivariate normal distributions and update the
estimates for \(\boldsymbol{\beta}\) and \(\phi\) via least squares in
each iteration of the EM algorithm. We create the augmented response,
augmented data, and augmented variance-covariance matrix following
\begin{align*}
& \boldsymbol{z}_* = \begin{bmatrix} \boldsymbol{z}\\ \boldsymbol{0}\end{bmatrix} &&
  \boldsymbol{X}_* = \begin{bmatrix} \boldsymbol{X} \\ \boldsymbol{I} \end{bmatrix} &&
  \boldsymbol{\Sigma}_* = \begin{bmatrix} \boldsymbol{\Sigma }& \boldsymbol{0}  \\ \boldsymbol{0} & \boldsymbol{\Sigma}_{\tau^2}/\phi \end{bmatrix}, &
\end{align*} such that \[
\boldsymbol{z}_* \sim \text{MVN}(\boldsymbol{X}_* \boldsymbol{\beta }, \phi \Sigma_*).
\] Using the least squares estimators to update \(\boldsymbol{\beta}\)
and \(\phi\), we have \begin{align*}
& \boldsymbol{\beta}^{(t)} = (\boldsymbol{X}_*^T \boldsymbol{\Sigma}^{-1} \boldsymbol{X}_*)^{-1}\boldsymbol{X}_*^T \boldsymbol{\Sigma}^{-1} \boldsymbol{z}_* && \phi^{(t)} = \frac{1}{n}(\boldsymbol{z}_*-X_*\boldsymbol{\beta}^{(t)})^T\boldsymbol{\Sigma}^{-1}(\boldsymbol{z}_*-X_*\boldsymbol{\beta}^{(t)}).&
\end{align*} To note, the variance-covariance matrix of the coefficient
estimates variance-covariance matrix can be derived in the EM-IWLS
algorithm and in turn can be used for statistical inferences, \[
  \text{Var}(\boldsymbol{\beta}^{(t)}) = (\boldsymbol{X}_*^T\boldsymbol{\Sigma}^{-1} \boldsymbol{X}_*)^{-1}\phi^{(t)}.
\]

Totally, the proposed EM-IWLS algorithm is summarized as follows:

\begin{enumerate}
\def\labelenumi{\arabic{enumi})}
\item
  Choose a starting value \(\boldsymbol{\beta}^{(0)}\) and
  \(\boldsymbol{\theta}^{(0)}\) for \(\boldsymbol{\beta}\) and
  \(\boldsymbol{\theta}\). For example, we can initialize
  \(\boldsymbol{\beta}^{(0)} = \boldsymbol{0}\) and
  \(\boldsymbol{\theta}^{(0)} = \boldsymbol{0}.5\)
\item
  Iterate over the E-step and M-step until convergence

  E-step: calculate \(E(\gamma_{j})\), \(E(\gamma^*_{j})\) and
  \(E(\tau^{-2}_{j})\), \(E({\tau^*}^{-2}_{jk})\) with the estimates
  \(\Theta^{(t-1)}\) from the previous iteration

  M-step:

  \begin{enumerate}
  \def\labelenumii{\alph{enumii})}
  \tightlist
  \item
    Based on the current value of \(\beta\), calculate the pseudo-data
    \(z_i^{(t)}\) and the pseudo-weights \(w_i^{(t)}\)
  \item
    Update \(\boldsymbol{\beta}^{(t)}\) by running the augmented
    weighted least squared
  \item
    If \(\phi\) is present, update \(\phi\)
  \end{enumerate}
\end{enumerate}

Similar to EM-CD, we assess convergence by the criterion,
\(|d^{(t)}-d^{(t-1)}|/(0.1+|d^{(t)}|)<\epsilon\), where \(\epsilon\) is
a small value (say \(10^{-5}\)).

\subsection{Supplementary Information 3: Generalization of BHAM framework}

In the manuscript, we describe the Bayesian hierarchical additive model
with the two-part spike-and-slab LASSO prior. Nevertheless, the proposed
model and algorithm can be easily generalized to accommodate other
priors thanks to the reparameterization of smoothing functions. To be
more specific, we can apply a regularized prior on the linear
coefficient, and a group regularized prior for the nonlinear
coefficients. For example, we can apply the same proposed framework with
a spike-and-slab mixture normal prior, \begin{align*}
  \beta_{j} |\gamma_{j},s_0,s_1 &\sim N(0,(1-\gamma_{j}) s_0 + \gamma_{j} s_1)\\
  \beta^*_{jk} | \gamma^*_{j},s_0,s_1 &\overset{\text{iid}}{\sim}N(0,(1-\gamma^*_{j}) s_0 + \gamma^*_{j} s_1), k=1,\dots, K_j.
\end{align*} The algorithm derivation still follows with slight
modification, replacing \(l_1\) penalization with \(l_2\) penalization.
The implementation of effect hierarchy would be more challenging for the
priors that do rely on a latent indicator for variable selection. As a
naive solution, readers can consider the linear prior and the nonlinear
prior being independent at the cost of bi-level selection accuracy.

\clearpage

\subsection{Supplementary Information 4: Predictive Performance of Linear Simulations}
\begin{table}[ht]
\centering
\begin{tabular}{cccccccc}
  \hline
P & mgcv & LASSO & COSSO & Adaptive COSSO & BHAM & SB-GAM & spikeSlabGAM \\ 
  \hline
  4 & 0.38 (0.01) & 0.39 (0.01) & 0.31 (0.08) & 0.29 (0.11) & 0.38 (0.01) & 0.35 (0.01) & 0.39 (0.01) \\ 
   10 & 0.36 (0.02) & 0.38 (0.01) & 0.35 (0.03) & 0.34 (0.04) & 0.39 (0.01) & 0.33 (0.02) & 0.39 (0.01) \\ 
   50 & 0.09 (0.09) & 0.37 (0.01) & 0.30 (0.06) & 0.30 (0.36) & 0.38 (0.01) & 0.32 (0.03) & 0.37 (0.01) \\ 
  100 & - & 0.37 (0.01) & 0.28 (0.07) & 0.34 (0.04) & 0.38 (0.01) & 0.29 (0.07) & 0.35 (0.01) \\ 
  200 & - & 0.36 (0.01) & 0.26 (0.08) & 0.31 (0.06) & 0.38 (0.03) & 0.28 (0.06) & 0.33 (0.02) \\ 
   \hline
\end{tabular}
\caption{The average and standard deviation of the out-of-sample $R^2$ measure for
    Gaussian outcomes over 50 iterations. The models of comparison include the proposed Bayesian
    hierarchical additive model (BHAM) fitted with Iterative Weighted Least Square (BHAM-IWLS) and
    Coordinate Descent (BHAM-CD) algorithms, component selection and smoothing operator (COSSO), adaptive
    COSSO, mgcv and sparse Bayesian generalized additive model (SB-GAM). mgcv doesn't provide estimation
    whe number of parameters exceeds sample size i.e. p = 100, 200.} 
\label{tab:lnr_gaus}
\end{table}

\begin{table}[ht]
\centering
\begin{tabular}{cccccccc}
  \hline
P & mgcv & LASSO & COSSO & Adaptive COSSO & BHAM & SB-GAM & spikeSlabGAM \\ 
  \hline
  4 & 0.79 (0.01) & 0.79 (0.01) & 0.76 (0.04) & 0.75 (0.04) & 0.78 (0.01) & 0.76 (0.01) & 0.79 (0.01) \\ 
   10 & 0.77 (0.01) & 0.79 (0.01) & 0.78 (0.01) & 0.78 (0.01) & 0.78 (0.01) & 0.75 (0.01) & 0.79 (0.01) \\ 
   50 & 0.62 (0.01) & 0.78 (0.01) & 0.75 (0.03) & 0.73 (0.04) & 0.74 (0.07) & 0.75 (0.02) & 0.77 (0.01) \\ 
  100 & - & 0.78 (0.01) & 0.73 (0.04) & 0.69 (0.05) & 0.73 (0.07) & 0.74 (0.02) & 0.76 (0.02) \\ 
  200 & - & 0.78 (0.01) & 0.71 (0.05) & 0.67 (0.05) & 0.73 (0.06) & 0.73 (0.03) & 0.72 (0.03) \\ 
   \hline
\end{tabular}
\caption{The average and standard deviation of the out-of-sample area under the curve measures
    for binomial outcomes over 50 iterations. The models of comparison include the proposed
    Bayesian hierarchical additive model (BHAM), linear LASSO model (LASSO), component
    selection and smoothing operator (COSSO), adaptive COSSO, mgcv, sparse Bayesian
    generalized additive model (SB-GAM), and spikeSlabGAM mdoel. mgcv doesn't provide
    estimation whe number of parameters exceeds sample size i.e. p = 100, 200.} 
\label{tab:lnr_bin_auc}
\end{table}


\clearpage

\subsection{Supplementary Information 5: Variable Selection Performance of Simulations}
\begin{table}[ht]
\centering
\begin{tabular}{ccccccccc}
  \hline
P & Metric & mgcv & LASSO & COSSO & Adaptive COSSO & BHAM & SB-GAM & spikeSlabGAM \\ 
  \hline
  4 & Precision & 1.00 & 1.00 & 1.00 & 1.00 & 1.00 & 1.00 & 1.00 \\ 
   10 & Precision & 0.88 & 0.58 & 0.71 & 0.69 & 0.93 & 0.88 & 0.49 \\ 
   50 & Precision &  & 0.38 & 0.60 & 0.59 & 0.77 & 0.80 & 0.04 \\ 
  100 & Precision &  & 0.35 & 0.62 & 0.65 & 0.82 & 0.77 & 0.02 \\ 
  200 & Precision &  & 0.29 & 0.65 & 0.57 & 0.21 & 0.74 & 0.03 \\ 
    4 & Recall & 1.00 & 0.54 & 0.61 & 0.54 & 0.38 & 0.97 & 0.55 \\ 
   10 & Recall & 0.92 & 0.46 & 0.50 & 0.54 & 0.38 & 0.97 & 0.31 \\ 
   50 & Recall & 0.00 & 0.34 & 0.30 & 0.30 & 0.32 & 0.90 & 0.07 \\ 
  100 & Recall &  & 0.27 & 0.30 & 0.25 & 0.27 & 0.92 & 0.06 \\ 
  200 & Recall &  & 0.27 & 0.25 & 0.29 & 0.56 & 0.92 & 0.05 \\ 
    4 & MCC &  &  &  &  &  &  &  \\ 
   10 & MCC & 0.89 & 0.18 & 0.35 & 0.36 & 0.47 & 0.86 & 0.10 \\ 
   50 & MCC &  & 0.25 & 0.34 & 0.36 & 0.45 & 0.83 & -0.04 \\ 
  100 & MCC &  & 0.24 & 0.37 & 0.37 & 0.45 & 0.82 & -0.01 \\ 
  200 & MCC &  & 0.23 & 0.36 & 0.36 & 0.23 & 0.81 & 0.01 \\ 
   \hline
\end{tabular}
\caption{} 
\label{}
\end{table}

\begin{table}[ht]
\centering
\begin{tabular}{ccccccccc}
  \hline
P & Metric & mgcv & LASSO & COSSO & Adaptive COSSO & BHAM & SB-GAM & spikeSlabGAM \\ 
  \hline
  4 & Precision & 1.00 & 1.00 & 1.00 & 1.00 & 1.00 & 1.00 & 1.00 \\ 
   10 & Precision & 0.91 & 0.59 & 0.97 & 0.97 & 0.90 & 0.56 & 0.74 \\ 
   50 & Precision & 0.39 & 0.43 & 0.74 & 0.84 & 0.89 & 0.39 & 0.25 \\ 
  100 & Precision &  & 0.34 & 0.53 & 0.72 & 0.91 & 0.29 & 0.25 \\ 
  200 & Precision &  & 0.27 & 0.42 & 0.52 & 0.92 & 0.33 & 0.25 \\ 
    4 & Recall & 1.00 & 1.00 & 0.87 & 0.84 & 1.00 & 1.00 & 1.00 \\ 
   10 & Recall & 1.00 & 1.00 & 0.98 & 0.96 & 1.00 & 1.00 & 0.75 \\ 
   50 & Recall & 1.00 & 1.00 & 0.85 & 0.99 & 1.00 & 1.00 & 0.25 \\ 
  100 & Recall &  & 1.00 & 0.84 & 0.97 & 1.00 & 1.00 & 0.25 \\ 
  200 & Recall &  & 1.00 & 0.78 & 0.88 & 1.00 & 0.98 & 0.25 \\ 
    4 & MCC &  &  &  &  &  &  &  \\ 
   10 & MCC & 0.91 & 0.59 & 0.96 & 0.94 & 0.91 & 0.54 & 0.58 \\ 
   50 & MCC & 0.57 & 0.59 & 0.76 & 0.90 & 0.92 & 0.57 & 0.18 \\ 
  100 & MCC &  & 0.54 & 0.61 & 0.81 & 0.94 & 0.48 & 0.22 \\ 
  200 & MCC &  & 0.49 & 0.52 & 0.64 & 0.95 & 0.52 & 0.23 \\ 
   \hline
\end{tabular}
\caption{} 
\label{}
\end{table}

\begin{table}[ht]
\centering
\begin{tabular}{cccccccc}
  \hline
P & Metric & LASSO & COSSO & Adaptive COSSO & BHAM & SB-GAM & spikeSlabGAM \\ 
  \hline
  4 & Precision & 1.00 & 1.00 & 1.00 & 1.00 & 1.00 & 1.00 \\ 
   10 & Precision & 0.61 & 0.97 & 0.98 & 0.55 & 0.74 & 0.91 \\ 
   50 & Precision & 0.35 & 0.59 & 0.68 & 0.26 & 0.59 & 0.61 \\ 
  100 & Precision & 0.28 & 0.47 & 0.51 & 0.30 & 0.46 & 0.57 \\ 
  200 & Precision & 0.26 & 0.44 & 0.42 & 0.38 & 0.41 & 0.38 \\ 
   \hline
  4 & Recall & 1.00 & 0.85 & 0.78 & 1.00 & 1.00 & 1.00 \\ 
   10 & Recall & 1.00 & 0.96 & 0.97 & 0.99 & 1.00 & 1.00 \\ 
   50 & Recall & 1.00 & 0.88 & 0.74 & 0.92 & 1.00 & 1.00 \\ 
  100 & Recall & 1.00 & 0.78 & 0.56 & 0.89 & 1.00 & 0.99 \\ 
  200 & Recall & 1.00 & 0.67 & 0.48 & 0.89 & 0.99 & 0.98 \\ 
   \hline
 10 & MCC & 0.58 & 0.95 & 0.96 & 0.49 & 0.72 & 0.91 \\ 
   50 & MCC & 0.52 & 0.66 & 0.65 & 0.37 & 0.73 & 0.74 \\ 
  100 & MCC & 0.49 & 0.55 & 0.47 & 0.43 & 0.65 & 0.72 \\ 
  200 & MCC & 0.49 & 0.49 & 0.39 & 0.47 & 0.62 & 0.58 \\ 
   \hline
\end{tabular}
\caption{The variable selection performance of linear binomial simulations,
                         measured by positive predictive value (precision), true positive rate (recall),
                         and Matthews correlation coefficient (MCC), for the high-dimensional methods
                         averaged over 50 iterations. The models of comparison include the proposed Bayesian
                         hierarchical additive model (BHAM), linear LASSO model (LASSO), component selection
                         and smoothing operator (COSSO), adaptive COSSO, sparse Bayesian generalized
                         additive model (SB-GAM), and spikeSlabGAM model. MCC is ill-defined when $p=4$
                         simulation (no true negative), and hence omitted for all methods.} 
\label{tab:sim_lnr_binom_var_select}
\end{table}


\clearpage
\subsection{References}

\hypertarget{refs}{}
\begin{CSLReferences}{1}{0}
\leavevmode\vadjust pre{\hypertarget{ref-Gelman2013}{}}%
Gelman, A, JB Carlin, HS Stern, DB Dunson, A Vehtari, and BD Rubin.
2013. {``Bayesian Data Analysis. 3rd Editio.''}

\leavevmode\vadjust pre{\hypertarget{ref-Yi2012}{}}%
Yi, Nengjun, and Shuangge Ma. 2012. {``{Hierarchical Shrinkage Priors
and Model Fitting for High-dimensional Generalized Linear Models}.''}
\emph{Statistical Applications in Genetics and Molecular Biology} 11
(6). \url{https://doi.org/10.1515/1544-6115.1803}.

\end{CSLReferences}

\end{document}
