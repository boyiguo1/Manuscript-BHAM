\documentclass[AMA,STIX1COL,]{WileyNJD-v2}


% Pandoc citation processing

\usepackage{tikz}
\usepackage{pgfplots}
\usepackage{multirow}
\usepackage{multicol}
\usepackage{colortbl}
\usepackage{hhline}
\usepackage{longtable}
\usepackage{array}
\usepackage{hyperref}

\articletype{Research article}

\received{2021-01-01}

\revised{2021-02-01}

\accepted{2021-03-01}

\raggedbottom

\providecommand{\tightlist}{%
  \setlength{\itemsep}{0pt}\setlength{\parskip}{0pt}}

\begin{document}

\title{Spike-and-Slab Generalized Additive Models and Scalable
Algorithms for High-Dimensional Data}

\author[a]{Boyi Guo*}
\author[b]{Byron C. Jaeger}
\author[a]{AKM Fazlur Rahman}
\author[a]{D. Leann Long}
\author[a]{Nengjun Yi*}

\authormark{Guo et al.}

\address[a]{Department of Biostatistics, University of Alabama at
Birmingham, Birmingham, USA}
\address[b]{Department of Biostatistics and Data Science, Wake Forest
University, Winston-Salem, USA}

\corres{Boyi Guo and Nengjun Yi, Department of Biostatistics, University
of Alabama at Birmingham, Birmingham, USA. \email{boyiguo1@uab.edu} and
\email{nyi@uab.edu}}

\presentaddress{This is sample for present address text this is sample
for present address text}

\abstract{There are proposals that extend the classical generalized
additive models (GAMs) to accommodate the high-dimensional
(p\textgreater\textgreater n) problem using group penalties and priors.
These models rarely smooth the nonlinear effect properly because of the
sparse regularization. It leads to excess shrinkage on coefficients and
inaccurate prediction. Besides, most of these GAMs consider an
``all-in-all-out'' approach for function selection, rendering them
difficult to answer if nonlinear effects are necessary. While some
Bayesian models can address the shortcomings, using Markov chain Monte
Carlo (MCMC) algorithms for model fitting creates a new challenge,
scalability. Hence, we propose Bayesian hierarchical generalized
additive models as a solution. For smooth and sparse solutions, the
proposed models employ bi-part spike-and-slab spline priors, including
mixture double exponential and mixture normal priors. Two
optimization-based algorithms, EM-Coordinate Descent and EM-Iterative
Weighted Least Squares, are developed for expedited computation.
Simulation studies and metabolomics data analyses demonstrate improved
performance against state-of-art models, mgcv, COSSO and sparse Bayesian
GAM. The software implementation of the proposed models is freely
available via an R package BHAM.}

\keywords{Spike-and-Slab Priors; High-Dimensional Data; Generalized
Additive Models; EM-IWLS; EM-Coordinate Decent}

\maketitle

\pgfplotsset{compat=1.18}
\usetikzlibrary{shapes.geometric, arrows, positioning, calc, matrix, backgrounds, fit}
\newcommand{\bs}[1]{\boldsymbol{#1}}
\newcommand{\tp}{*}
\newcommand{\pr}{\text{Pr}}

\hypertarget{introduction}{%
\section{Introduction}\label{introduction}}

\label{sec:intro}

Many modern biomedical research, e.g.~sequencing data analysis, electric
health record data analysis, require special treatment of
high-dimensionality, commonly known as \(p >> n\) problem. There is
extensive literature on high-dimensional linear models via penalized
models or Bayesian hierarchical models, see Mallick and Yi
\citep{Mallick2013} for review. These models are built upon a
restrictive and unrealistic assumption, linearity. In classical
statistical modeling, many strategies and models are proposed to relax
the linearity assumption with various degrees of complexity. For
example, variable categorization is a simple and common practice in
epidemiology, but suffers from power and interpretation issues. More
complex models to address nonlinear effects include random forest and
other so-called ``black box'' models \citep{Breiman2001}. These models
are not particularly attractive in biomedical studies, as they lack the
transparency to understand the etiology of diseases. In addition, how to
generalized these models to the high-dimensional setting remains
unclear.

Nonparametric regression models are appropriate alternatives to the
``black-box'' models thanks to their balance between model flexibility
and interpretability. Among those, generalized additive models (GAMs),
proposed in the seminal work of Hastie and Tibshirani
\citep{Hastie1987}, grew to be one of the most popular modeling tools.
In a GAM, the response variable, \(Y\), which is assumed to follow some
exponential family distribution \(EF(\mu, \phi)\) with mean \(\mu\) and
dispersion \(\phi\), can be expressed as the summation of smoothing
functions, \(B_j, j = 1, \dots, p\), of a given \(p\)-dimensional vector
of covariates \(\boldsymbol{x}\), written as \[
 E(Y|\boldsymbol{x}) = g^{-1}(a + \sum\limits^p_{j=1}B_j(x_j)),
\] where \(g^{-1}\) is the inverse of a monotonic link function. The
smoothing functions can take many forms and are estimated using a
pseudo-weighted version of the backfitting algorithm
\citep{Breiman1985}. Nevertheless, the classical GAMs cannot fulfill the
increasing analytic demands for high-dimensional data analysis in
biomedical studies.

There exists some proposals to generalize the classical GAM to
accommodate high-dimensional applications. The regularized models,
branching out from group regularized linear models, are used to fit GAMs
by accounting for the structure introduced when expanding smoothing
functions. Ravikumar et al. \citep{Ravikumar2009} extended the grouped
lasso \citep{Yuan2006} to additive models (AMs); Huang et al.
\citep{Huang2010} further developed adaptive grouped lasso for additive
models; Wang et al. \citep{Wang2007} and Xue \citep{Xue2009}
respectively applied grouped SCAD penalty \citep{Fan2001} to additive
models. Recent Bayesian hierarchical models are also used in the context
of high-dimensional additive models. Various group spike-and-slab priors
combining with computationally intensive Markov chain Monte Carlo (MCMC)
algorithms \citep{Xu2015, Yang2020} are proposed, where the application
on AMs are by-products. Bai et al. \citep{Bai2020} was the first to
apply group spike-and-slab lasso prior to Gaussian AMs using a fast
optimization algorithm, and further generalized the framework to
GAMs\citep{Bai2021}. As these methods focus on addressing the sparsity,
it is very likely to overly penalize the basis function coefficients and
produce inaccurate predictions and curve interpolation, particularly
when complex signals are assumed and large number of knots are used.
\citep{Scheipl2013} In addition, these methods adapt an `all-in-all-out'
strategy, i.e.~either including or excluding the variable completely,
rendering no space for bi-level selection. Scheipl et al.
\citep{Scheipl2012} proposed a spike-and-slab structure prior,
normal-mixture-of-inverse gamma prior, that address the previous
challenges. But the model fitting relies on computational intensive MCMC
algorithmic and creates scalability concern. It would be of special
interest to develop a fast, flexible and accurate generalized additive
model framework.

To address these challenges, we propose a novel Bayesian hierarchical
generalized additive model (BHAM) for high dimensional data analysis.
Specifically, we incorporate smoothing penalties in the model via
re-parameterization of the smoothing function to avoid overly shrinking
basis function coefficients. Smoothing penalties are commonly
implemented in the classical GAMs through smoothing regression splines.
They are quadratic norms of the coefficients and allow locally adaptive
penalties on each smoothing function. A smoothing penalty conditioning
on a smoothing parameter \(\lambda_j\) is a function of the integration
of the second derivative of the spline function, expressed
mathematically as \begin{equation}\label{eq:smoothpen}
  \text{pen}\left[f_j(x)\right] = \lambda_j \int f^{\prime\prime}_j(x)dx = \lambda_j \boldsymbol{\beta}_j^T \boldsymbol{S}_j \boldsymbol{\beta}_j ,
\end{equation} where \(\boldsymbol{S}_j\) is a known smoothing penalty
matrix, and \(\boldsymbol{\beta}_j\) are the basis function
coefficients. Smoothing penalties were also previously used in the
spike-and-slab GAM \citep{Scheipl2012} and the sparsity-smoothness
penalty \citep{Meier2009}. Moreover, incorporating the smoothing penalty
allows the separation of the linear space of smoothing functions, also
know as the null space, from the nonlinear space. We then impose a new
bi-part spike-and-slab spline prior on the smoothing functions for
bi-level selection such that the linear and nonlinear spaces of
smoothing functions can be selected separately. The prior setup
encourages a flexible solution, rending one of three possibilities: no
effect, only linear effect, or nonlinear effect of a predictor. In
addition, two fast computing optimization-based algorithms are developed
for high and ultrahigh-dimensional\citep{Fan2009} settings respectively.

The proposed framework, BHAM, differs from previous spike-and-slab based
GAMs, i.e.~the spike-and-slab GAM \citep{Scheipl2012} and the SB-GAM
\citep{Bai2021} in many ways. First of all, the proposed spike-and-slab
spline prior is a spike-and-slab lasso type prior using independent
mixture double exponential distribution, compared to spike-and-slab GAM.
Spike-and-slab lasso priors provide computational convenience during
model fitting by using optimization algorithms instead of computational
intensive sampling algorithms. They make fitting high- and
ultrahigh-dimensional models more feasible without sacrificing
performance in prediction and variable selection. Secondly, the
optimization algorithms introduced in this article include EM-Coordinate
Descent (EM-CD) algorithm and EM-Iterative weighted least square
(EM-IWLS) algorithm for different utilities. The applications of the two
aforementioned algorithm are fruitful in Bayesian hierarchical models in
high-dimensional data analysis.
\citep{Yi2012, Rockova2014a, Rockova2018} Even though SB-GAM with a
group spike-and-slab lasso prior also uses EM-CD for model fitting, the
proposed EM-IWLS provides uncertainty measures that could be further
used for confidence bound calculation and hypothesis testing. EM-IWLS
complements the lack of inference potential from EM-CD in a median and
high dimensional setting. Furthermore, the proposed model addresses the
incapability of bi-level selection in SB-GAM. Lastly, the R
implementation of the Bayesian hierarchical additive model is available
publicly via BHAM at \url{https://github.com/boyiguo1/BHAM}, make
transnational science more accessible.

In Section \ref{sec:BHAM}, we establish the Bayesian hierarchical
generalized additive model, introduce the proposed spike-and-slab spline
priors, and describe the two fast-fitting algorithms. In Section
\ref{sec:sim}, we compare the proposed framework to state-of-art models,
mgcv, COSSO and sparse Bayesian GAM via Monte Carlo simulation studies.
Analyses of two metabolomics datasets are presented in Section
\ref{sec:real_data}. Conclusion and discussions are given in Section
\ref{sec:concl}.

\hypertarget{bayesian-hierarchical-additive-models}{%
\section{Bayesian Hierarchical Additive
Models}\label{bayesian-hierarchical-additive-models}}

\label{sec:BHAM}

Following the GLM notation introduced in Section \ref{sec:intro}, we
have a generalized additive model and its matrix form,
\begin{equation}\label{eq:gam}
E(Y|\boldsymbol{x}) = g^{-1}(a + \sum\limits^p_{j=1}B_j(x_j)) = g^{-1}(a + \sum\limits^p_{j=1} \boldsymbol{\beta}_j^T \boldsymbol{X}_j),
\end{equation} and the data distribution is \begin{equation}
f(y|X\beta, \phi) = \prod\limits^n_{i=1}f(y_i|X_i,\phi),\nonumber
\end{equation} with smoothing functions \(B_j(x_j)\) of the variable
\(x_j, j = 1, \dots, p.\) The basis function matrix, i.e.~the design
matrix derived from the smoothing function \(B_j(x_j)\), is denoted
\(\boldsymbol{X}_j\) for the variable \(x_j\). The dimension of the
design matrix depends on the choice of the smoothing function. For
example, a cubic spline function with \(k\) knots produces a \(k+3\)
elements column vector. Its corresponding smoothing penalty matrix is
denoted as \(\boldsymbol{S}_j\). \(\boldsymbol{\beta}_j\) denotes the
basis function coefficients for the \(j\)th variable, and the matrix
transposing operation is denoted with a superscript \(^T\). To note, the
proposed model can include parametric forms of variables in the model,
treating the parametric function as a special case of the smoothing
function, e.g.~\(B_j(x_j) = x_j\) with the smoothing penalty matrix
defined as \(\boldsymbol{S}_j = \begin{bmatrix}0\end{bmatrix}\).

To encourage proper smoothing of the functions, we adopt the idea of
smoothing penalties from smoothing spline models. The basic idea is to
set up a smoothing penalty, described in Equation (\ref{eq:smoothpen}),
in the prior density function, or least squared and likelihood function
in the penalized framework. However, the direct integration of smoothing
penalties into sparsity penalties is not obvious. Marra and Wood
\citep{Marra2011} proposed a re-parameterization procedure to
accommodate the smoothing penalty implicitly. Given the smoothing
penalty matrix is most often symmetric and positive semi-definite for
univariate smoothing functions, we apply eigendecomposition on the
penalty matrix
\(\boldsymbol{S} = \boldsymbol{U} \boldsymbol{D} \boldsymbol{U}^T\) ,
where the matrix \(\boldsymbol{D}\) is diagonal with the eigenvalues
arranged in the ascending order. To note, \(\boldsymbol{D}\) can contain
\(0\) elements on the diagonal, associated with the linear space of the
smoothing function. We further write the orthonormal matrix
\(\boldsymbol{U} \equiv \begin{bmatrix}\boldsymbol{U}^0 : \boldsymbol{U}^{*}\end{bmatrix}\)
containing the eigenvectors as columns in the corresponding order. That
is \(\boldsymbol{U}^0\) contains the eigenvectors with zero eigenvalues
for the linear space and \(\boldsymbol{U}^{*}\) contains the
eigenvectors (as columns) for the non-zero eigenvalues, i.e.~the
non-linear space. We multiply the basis function matrix
\(\boldsymbol{X}\) with the orthonormal matrix \(\boldsymbol{U}\) for
the new design matrix
\(\boldsymbol{X}^\ast = \boldsymbol{X} \boldsymbol{U} \equiv \begin{bmatrix} \boldsymbol{X}^0 : \boldsymbol{X}^{*} \end{bmatrix}\).
An additional scaling step is imposed on \(\boldsymbol{X}^{*}\) by the
non-zero eigenvalues of \(\boldsymbol{D}\) such that the new basis
function matrix \(\boldsymbol{X}^\ast\) can receive uniform penalty on
each of its dimensions. With slight abuse of the notation, we denote
\(\boldsymbol{X}_j \equiv \begin{bmatrix} \boldsymbol{X}_j^0 : \boldsymbol{X}_j^{*} \end{bmatrix}\)
as the basis function matrix for the \(j\)th variable after the
re-parameterization. A spline function can be expressed in the matrix
form \[
B_j(x_j) = B_j^0(x_j) + B_j^*(x_j) = \boldsymbol{\beta_j^0}^T \boldsymbol{X}_j^0 + \boldsymbol{\beta_j^*}^T \boldsymbol{X}_j^*,
\] and the generalized additive model in Equation (\ref{eq:gam}) be
\begin{equation}\label{eq:gam-repa}
E(Y|\boldsymbol{x}) =  g^{-1}(a + \sum\limits^p_{j=1} \boldsymbol{\beta}_j^T \boldsymbol{X}_j) = g^{-1}\left[a + \sum\limits^p_{j=1} (\boldsymbol{\beta_j^0}^T \boldsymbol{X}_j^0 + {\boldsymbol{\beta}_j^*}^T \boldsymbol{X}_j^*)\right].
\end{equation}

The re-parameterization step sets up the foundation of the proposed
model and provides three-fold benefits. First of all, the
re-parameterization integrates the smoothing penalty into the design
matrix, and encourages models to properly smooth the nonlinear function
in addition to the sparse penalty for function selection. Secondly, the
eigendecomposition of the smoothing penalty allows the isolation of the
linear from the nonlinear space, improving the feasibility of bi-level
function selection and inference. The eigendecomposition facilitates the
construction of orthonormal design matrix, which makes imposing
independent priors on the coefficients possible. This reduces the
computational complexity compared to using a multivariate priors, and
greatly broadens the choices of priors and further model choices. Last
but not least, the scaling step of \(\boldsymbol{X}^*\) further
simplifies the choice of priors, from column-specific priors to a
unified prior.

\hypertarget{spike-and-slab-smooth-priors}{%
\subsection{Spike-and-Slab Smooth
Priors}\label{spike-and-slab-smooth-priors}}

The family of spike-and-slab (SS) priors regression models dominates
Bayesian high-dimensional analysis for its utility in outcome prediction
and variable selection. We defer to Bai et al. \citep{Bai2021Review} for
an in-depth introduction to spike-and-slab priors. To summarize,
spike-and-slab priors are a family of mixture distributions that
comprises a skinny spike density \(f_{\text{spike}}(\cdot)\) for weak
signals and a flat slab density \(f_{\text{slab}}(\cdot)\) for strong
signals, mathematically \[
 \beta|\gamma \sim (1-\gamma)f_{\text{spike}}(\beta) + \gamma f_{\text{slab}}(\beta).
\] The most distinct feature of SS priors is that it is conditioned on a
latent binary variable \(\gamma \in \{0,1\}\) that indicates whether the
variable \(x\) is included in the model. There are various
spike-and-slab priors depending on the choice for the spike density
\(f_{\text{spike}}(\cdot)\) and the slab density
\(f_{\text{slab}}(\cdot)\), see George and McCulloch
\citep{George1993, George1997}; Chipman \citep{Chipman1996} for grouped
variables; Brown et al. \citep{Brown1998} for multivariate outcomes;
Ishwaran and Rao \citep{Ishwaran2005}; Clyde and George
\citep{Clyde2004} and reference therein.

The major criticism of early spike-and-slab priors receive is being
computationally prohibitive. \citep{Bai2021Review} Since then, many
studies focus on alleviating the computational burden that sampling
algorithms bear, which include EMVS \citep{Rockova2014a} and
Spike-and-slab Lasso \citep{Rockova2018b, Rockova2018}. Particularly,
the development of Spike-and-slab Lasso substantially improves the
scalability of SS models, setting up the theoretical foundation for
generalized models in -omics data analysis
\citep{Tang2017a, Tang2017, Tang2018, Tang2019}. The spike-and-slab
lasso prior is composed of two double exponential distributions with
mean 0 and different dispersion parameters, \(0 < s_0 < s_1\),
mathematically, \begin{equation} 
\beta | \gamma \sim (1-\gamma)DE(0, s_0) + \gamma DE(0, s_1), 0 < s_0 < s_1.\nonumber
\end{equation} Given that both double exponential distributions have a
mean of 0 and the latent indicator \(\gamma\) can only take the value of
0 or 1, the mixture double exponential distribution can be formulated as
a double exponential density, \begin{equation} \label{eq:ssl}
\beta | \gamma \sim DE(0, (1-\gamma^0_{j}) s_0 + \gamma^0_{j} s_1), 0 < s_0 < s_1,
\end{equation} with the scale parameter
\(S = (1-\gamma)s_0 + \gamma s_1\). The SSL also mitigates the problem
of EMVS where the weak signals are not shrink to zero, and hence is
preferred in high-dimensional data analysis. We notice that Bai
\citep{Bai2021} is the first to apply spike-and-slab lasso prior in the
GAM framework, where the densities of the spike and slab components take
the group lasso density \citep{Xu2015}. Bai \citep{Bai2021} takes an
``all-in-all-out'' strategy for function selection.

\hypertarget{spike-and-slab-lasso-prior}{%
\subsubsection{Spike-and-Slab Lasso
Prior}\label{spike-and-slab-lasso-prior}}

We introduce a novel prior for GAMs, particularly for high-dimensional
nonlinear modeling with bi-level selection. The proposed prior extends
from the spike-and-slab lasso prior described in Equation
(\ref{eq:ssl}). Given the re-parameterized deign matrix
\(\boldsymbol{X}_j = \begin{bmatrix} \boldsymbol{X}_j^0 : \boldsymbol{X}_j^*\end{bmatrix}\)
for the \(j\)th variable, we impose a two-part SSL prior to the
coefficients \(\boldsymbol{\beta}_j\). Specifically, we impose
independent group priors on the linear space coefficients and on the
nonlinear space coefficients respectively,

\begin{align}\label{eq:bham_ssl}
  \beta^0_{j} |\gamma^0_{j},s_0,s_1 &\sim DE(0,(1-\gamma^0_{j}) s_0 + \gamma^0_{j} s_1) \nonumber \\
  \beta^*_{jk} | \gamma^*_{j},s_0,s_1 &\sim DE(0,(1-\gamma^*_{j}) s_0 + \gamma^*_{j} s_1), 
\end{align}

where \(\gamma^0_{j}\in\{0,1\}\) and \(\gamma^*_{j}\in \{0,1\}^{K_j}\)
are latent indicator variable and vectors, indicating if the model
includes the linear effect and the nonlinear effect of the \(j\)th
variable respectively. \(s_0\) and \(s_1\) are scale parameters,
assuming \(0 < s_0 < s_1\) and given. These scale parameters \(s_0\) and
\(s_1\) can be treated as tuning parameters and optimized via
cross-validation. A discussion of how to choose the scale parameters
comes in Section \ref{sec:tune}. To note, this prior differs from
previous group spike-and-slab lasso priors \citep{Tang2018, Tang2019},
where the \(\boldsymbol{\beta}^0_j\) and \(\boldsymbol{\beta}^*_j\)
shares the same indicator variables \(\gamma_j^0\), \(\gamma_j^*\)
respectively. It is possible to add a more restrictive assumption on the
priors, assuming that one indicator variable decides the inclusion of
both the linear effect and nonlinear effect,
i.e.~\(\gamma_j = \gamma^0_j = \gamma^*_j\). This converges to the
SB-GAM \citep{Bai2021}. Conversely, it is also possible to relax the
assumption such that each coefficient \(\beta\) to have its own latent
indicator \(\gamma\), but at the cost of complicating the bi-level
function selection. This reduces the proposed prior to the classic
spike-and-slab lasso prior.

The re-parameterization introduced in Section \ref{sec:BHAM} grants the
validity of the proposed prior. First of all, the smoothing function
bases are linear dependent and ask for extra attention. The
eigeondecomposition remedies the problem and hence our prior can be set
to be conditionally independent. Secondly, the eigenvalue scaling
provides a panacea to allow unified scale parameters for all bases of
all smoothing functions of variables.

The rest of the hierarchical prior follows the traditional
spike-and-slab lasso prior: we set up hyper-priors of
\(\boldsymbol{\gamma}\) to allow local adaption of the shrinkage using a
Bernoulli distribution (binomial distribution of one trial). The two
indicators of the \(j\)th predictor, \(\gamma^{0}_j\) and
\(\gamma^*_j\), shares the same probability parameter \(\theta_j\), \[
\begin{aligned}
\gamma_{j}^{0} | \theta_j &\sim Bin(\gamma^{0}_{j}|1, \theta_j)\\
\gamma_{j}^*| \theta_j &\sim Bin(\gamma^*_{j}|1, \theta_j).
\end{aligned}
\] This is to leverage the fact that the probability of selecting the
bases of a smoothing function should be similar, while allowing
different penalty on the linear space and non-linear space of the
smoothing function. The hyper prior of \(\gamma_{j}^{0}\) decides the
sparsity of the model at the function selection level, while that of
\(\gamma_{j}^*\) decides the smoothness of the spline function at basis
function level. Meanwhile, we specify that \(\gamma_{j}^0\) and
\(\gamma_{j}^*\) are independently distributed for analytic simplicity.
We further specify the parameter \(\theta_j\) follows a beta
distribution with given shape parameters \(a\) and \(b\), \[
\theta_j \sim Beta(a, b).
\] The beta distribution is a conjugate prior for the binomial
distribution and hence provides some computation convenience. For
simplicity, we focus on a special case of beta distribution, uniform
(0,1). When the variable have large effects in any of the bases, the
parameter \(\theta_j\) will be estimated large, which in turn encourages
the model to include the rest of bases, achieving the local adaption
among spline bases. Hereafter, we refer Bayesian hierarchical
generalized additive models with the spike-and-slab lasso prior as the
BHAM-SSL, and visually presented in Figure \ref{fig:SSprior}.

\begin{figure}
\centering
\begin{tikzpicture} [
staticCompo/.style = {rectangle, minimum width=1cm, minimum height=1cm,text centered, draw=black, fill=blue!30},
outCome/.style={ellipse, minimum width=3cm, minimum height=1cm,text centered, draw=black, fill=blue!30},
mymatrix/.style={matrix of nodes, nodes=outCome, row sep=1em},
PriorBoarder/.style={rectangle, minimum width=5cm, minimum height=10cm, text centered, fill=lightgray!30},
background/.style={rectangle, fill=gray!10,inner sep=0.2cm, rounded corners=5mm}
]

\matrix (linearPrior) [matrix of nodes, column sep = 0mm, row sep = 0.7cm] {
  \node (linearGamma) [outCome] { $\gamma_j^0 \sim Bin(1, \theta_j) $ };\\
  \node (linearBeta) [outCome] { $\beta_j^0 \sim DE(0,(1-\gamma^0_{j}) s_0 + \gamma^0_{j} s_1)$};\\
};
\matrix (penPrior) [right = 2cm of linearPrior, matrix of nodes, column sep = 0mm, row sep = 0.7cm] {
  \node (penGamma) [outCome] { $\gamma_{j}^*\sim Bin(1, \theta_j)$ };\\
  \node (penBeta) [outCome] { $\beta_{jk}^*\sim  DE(0,(1-\gamma^*_{j}) s_0 + \gamma^*_{j} s_1)$};\\
};


\node (s) [staticCompo]  at ($(linearBeta)!0.5!(penBeta)$)  {($s_0, s_1$)};
\node (Beta) [staticCompo, below = 1cm of s] {$\boldsymbol{\beta }= (\beta^0_1, \boldsymbol{\beta}^*_1, \dots,\beta^0_j, \boldsymbol{\beta}^*_j , \dots,\beta^0_p, \boldsymbol{\beta}^*_p) $};
\node (Theta)[outCome, above = 2cm of s] {$\theta_{j} \sim Beta(a, b)$};
\node (ab)[staticCompo, above = 0.5cm of Theta] {$(a, b)$};
\node (Y) [outCome, below = 1cm of Beta] {$y_i \sim Expo. Fam. (g^{-1}(\boldsymbol{\beta}^T \boldsymbol{X}_i), \phi)$};

\draw[->] (Theta) -- (linearGamma);
\draw[->] (Theta) -- (penGamma);
\draw[->] (linearGamma) -- (linearBeta) ;
\draw[->] (penGamma) -- (penBeta);
\draw[->] (ab) -- (Theta);
\draw[->] (s) -- (linearBeta) ;
\draw[->] (s) -- (penBeta);
\draw[->] (linearBeta) -- (Beta);
\draw[->] (penBeta) -- (Beta);
\draw[->] (Beta) --  (Y);


\begin{pgfonlayer}{background}
  \node [background,
   fit=(linearGamma) (linearBeta),
   label=above:Linear Space:] {};
  \node [background,
    fit=(penGamma) (penBeta),
    label=above:Nonlinear Space:] {};
\end{pgfonlayer}

\end{tikzpicture}

\caption{Directed acyclic grpah of BHAM-SSL model with parameter expansion. Elliposes are stochastic nodes, rectangles and are deterministic nodes. }
\label{fig:SSprior}
\end{figure}

\hypertarget{other-priors}{%
\subsubsection{Other Priors}\label{other-priors}}

With the re-parameterization step of the basis function matrix \(X\), it
is possible to generalized the SSL prior to other priors, for example
normal priors for ridge-type regularization and mixture normal prior for
spike-and-slab regularization. These priors would work better in low and
medium dimensional settings where the sparse assumption is not
necessary. Here we focus on the mixture normal prior as a demonstration
of applying continuous spike-and-slab prior in additive models.

A spike-and-slab mixture normal prior in additive models can be
expressed as \[
\begin{aligned}
  \beta^0_{j} |\gamma^0_{j},s_0,s_1 &\sim N(0,(1-\gamma^0_{j}) s_0 + \gamma^0_{j} s_1)\nonumber\\
  \beta^*_{jk} | \gamma^*_{j},s_0,s_1 &\sim N(0,(1-\gamma^*_{j}) s_0 + \gamma^*_{j} s_1).
\end{aligned}
\] Similar to the BHAM-SSL prior in Equation (\ref{eq:bham_ssl}),
\(0 < s_0 < s_1\) are tuning parameters and can be optimized via
cross-validation. One of the critics received by the spike-and-slab
mixture normal prior is that the tails of a normal distribution
diminishes to zero too fast, which renders problem when estimating the
large effects. Distributions with heavier tails can be used as an
alternative, for example mixture \(t\) distribution with small degree of
freedom.

\hypertarget{algorithms-for-fitting-bhams}{%
\subsection{Algorithms for Fitting
BHAMs}\label{algorithms-for-fitting-bhams}}

The proposed models can be fitted with MCMC algorithms. Nevertheless,
the computational burden of MCMC algorithms creates scalability issues.
George and McCulloch\citep{George1997} examined the computation speed
for various MCMC algorithms with continous spike-and-slab priors, and
suggested MCMC algorithms works well for medium size (\(p\)=25) of
predictors with only linear effects . However, it is not feasible to
high-dimensional data analysis where the number of predictors easily
exceeds 100. Specific to additive models, each predictor would expand to
multiple new ``predictors'' via basis functions, creating greater
computational demands. Scheipl et al. \citep{Scheipl2013} demonstrated
the computational demands of a MCMC algorithm for fitting spike-and-slab
additive models grow exponentially as \(p\) increases modestly via monte
carlo simulation studies. Hence, we feel compelled to develop scalable
algorithms for fitting Bayesian hierarchical additive models in
high-dimensional settings.

As alternatives to the sampling algorithms for fitting Bayesian models,
optimization algorithms focus on the maximum a posteriori (MAP)
estimates and speed up the model fitting process at the cost of
uncertainty measures. The earlier work for fitting spike-and-slab models
using optimization algorithms includes EMVS \citep{Rockova2014a}.
Rockova and George\citep{Rockova2014a} proposed an
expectation-maximization (EM) based algorithm to fit models that use
continuous mixture normal priors. In the E step, the latent binary
indicators \(\gamma\)s are treated as the missing data, and the
posterior means are calculated conditioning on the current value of
other parameters; in the M step, a ridge estimator was used to update
the coefficients, and \(\sigma^2, \theta\) are updated accordingly. The
same authors \citep{Rockova2018b, Rockova2018} further combined the EM
algorithm with coordinate descent algorithm for fitting SSL models. Yi
and his group independently developed the EM-Iterative Weighted Least
Square and EM-cyclic coordinate descent algorithms to fit models with
broader class of priors, SSL included.\citep{Yi2019} These algorithms
were implemented for generalized linear models\citep{Tang2017a}, Cox
proportional hazards models \citep{Tang2017} and their grouped
counterparts\citep{Tang2018, Tang2019}. Both EM based algorithms provide
deterministic solutions, which becomes a more popular property for
reproducible research

In the follow section, we expand the two EM-based algorithms, EM-CD and
EM-IWLS algorithms, to fit BHAMs. To note, the two proposed algorithms
provides different utilities. The EM-CD algorithm is specifically for
fitting BHAM-SSL with an expedited performance, recommending to use in
high and ultra-high dimensional setting. SB-GAM\citep{Bai2020, Bai2021}
also used EM-CD algorithm. The main difference between proposed EM-CD
algorithm and SB-GAM is that SB-GAM uses a blocked CD algorithm for
their group prior. In comparison, the proposed prior is pairwise
independent requiring no special treatment. A specific concern of EM-CD
algorithm is that it provides no information for inference. In contrast,
the EM-IWLS can extract the variance-covariance matrix thanks to the
IWLS algorithm and construct Wald type inference. Moreover, the EM-IWLS
is a more general model fitting algorithm that can be used for fitting
not only SSL and continuous SS priors but also Student's t-priors and
double exponential priors.

\hypertarget{em-algorithms}{%
\subsubsection{EM algorithms}\label{em-algorithms}}

EM algorithm is an iterative algorithm to find MAP estimates or the
maximum likelihood estimates. It is commonly used when some necessary
data to establish the likelihood function are missing. Instead, the
algorithm maximizes the expectation, with respect to the ``missing''
data, of the likelihood function.

The recursive algorithm consists of two steps:

\begin{itemize}
\tightlist
\item
  E-step: to calculate the expectation of the likelihood function
  conditioning on some ``missing'' data
\item
  M-step: to maximize the expected likelihood function to calculate the
  parameter of interest
\end{itemize}

For BHAMs, we define the parameters of interest as
\(\Theta = \{\boldsymbol{\beta}, \boldsymbol{\theta}, \phi\}\) and
consider the latent binary indicators \(\boldsymbol{\gamma}\) as
nuisance parameters of the model. Our objective is to find the
parameters \(\Theta\) that maximize the posterior density function, or
equivalently, the logarithm of the density function, \[
\begin{aligned}
& \text{argmax}_{\Theta}
\log f(\Theta, \boldsymbol{\gamma}| \textbf{y}, \textbf{X}) \\
&= \log f(\textbf{y}|\boldsymbol{\beta}, \phi) + \log f(\phi) + \sum\limits_{j=1}^p\left[\log f(\beta^0_j|\gamma^0_j)+\sum\limits_{k=1}^{K_j} \log f(\beta^{*}_{jk}|\gamma^{*}_{jk})\right]\\
& +\sum\limits_{j=1}^{p} \left[ (\gamma^0_j+\gamma_{j}^{*})\log \theta_j + (2-\gamma^0_j-\gamma_{j}^{*}) \log (1-\theta_j)\right] +  \sum\limits_{j=1}^{p}\log f(\theta_j)
\end{aligned}
\]\\
For simplicity, we consider the linear space to be 1-dimension, which
holds true for many smoothing functions.

We use the EM algorithm to find the MAP estimate of \(\Theta\). Since
the latent binary indicators \(\boldsymbol{\gamma}\) are not of our
primary interest, we treat them as the ``missing data'' in the EM
algorithm. Naturally, in the E-step, we calculate the expectation of
posterior density function of
\(\log f(\Theta, \boldsymbol{\gamma}| \textbf{y}, \textbf{X})\) with
respect to the latent indicators \(\boldsymbol{\gamma}\) conditioning on
the values from previous iteration \(\Theta^{(t-1)}\), \[
E_{\boldsymbol{\gamma}|\Theta^{(t-1)}}\log f(\Theta, \boldsymbol{\gamma}| \textbf{y}, \textbf{X}) .
\] Hereafter, we use the shorthand notation
\(E(\cdot)\equiv E_{\boldsymbol{\gamma}|\Theta^{(t-1)}}(\cdot)\). In the
M-step, we find the \(\Theta^{(t)}\) that maximize
\(E\log f(\Theta, \boldsymbol{\gamma}| \textbf{y}, \textbf{X})\). To
note here, the log-posterior density of BHAMs (up to additive constants)
can be written as a two-part equation

\[ \log f(\Theta, \boldsymbol{\gamma}| \textbf{y}, \textbf{X}) = Q_1(\boldsymbol{\beta}, \phi) + Q_2 (\boldsymbol{\gamma},\boldsymbol{\theta}),\]
Where
\[ Q_1(\boldsymbol{\beta}, \phi) = \log f(\textbf{y}|\boldsymbol{\beta}, \phi) + \log f(\phi) + \sum\limits_{j=1}^p\left[\log f(\beta^0_j|\gamma^0_j)+\sum\limits_{k=1}^{K_j} \log f(\beta^{*}_{jk}|\gamma^{*}_{jk})\right]\]
and \[
Q_2(\boldsymbol{\gamma},\boldsymbol{\theta}) = \sum\limits_{j=1}^{p} \left[ (\gamma^0_j+\gamma_{j}^{*})\log \theta_j + (2-\gamma^0_j-\gamma_{j}^{*}) \log (1-\theta_j)\right] +  \sum\limits_{j=1}^{p}\log f(\theta_j) .\]
\(Q_1\) and \(Q_2\) are, respectively, the log-likelihoods of model
coefficients and the dispersion parameter \(\boldsymbol{\beta}, \phi\)
and latent indicators \(\boldsymbol{\gamma}, \theta\). Conditioning on
\(\boldsymbol{\gamma}\), \(Q_1\) and \(Q_2\) are independent and can be
maximized separately for \(\boldsymbol{\beta}, \phi\) and
\(\boldsymbol{\theta}\).

The E- and M- steps are iterated until the algorithm converge. Depending
on the choice of the mixture priors, the maximization in the M-step can
take different algorithms, CD for fast computation and IWLS for
uncertainty measures.

\hypertarget{em-coordinate-descent}{%
\subsubsection{EM-Coordinate Descent}\label{em-coordinate-descent}}

When the prior distribution of the coefficients is set to mixture double
exponential, coordinate descent algorithm can be used to estimate the
parameters in the M-step. Coordinate descent is an optimization
algorithm that offers extreme computational advantage, and famous for
its application in optimizing the \(l_1\) penalized likelihood function.

To recall, the density function of spike-and-slab mixture double
exponential prior can be written as \[
f(\beta|\gamma, s_0, s_1) = \frac{1}{2\left[(1-\gamma)s_0 + \gamma s_1\right]}\exp(-\frac{|\beta|}{(1-\gamma)s_0 + \gamma s_1}),
\] and \(E(Q_1)\) can be expressed as a likelihood function with \(l_1\)
penalty \begin{equation}\label{eq:Q1_CD}
E(Q_1) = \log f(\textbf{y}|\boldsymbol{\beta}, \phi) + \log f(\phi) - \sum\limits_{j=1}^p\left[E({S^0_j}^{-1})|\beta^0_j|+\sum\limits_{k=1}^{K_j}E(S^{-1}_{j})|\beta_{jk}|\right],
\end{equation} where
\(S_{j}^0 = (1-\gamma^{0}_{j}) s_0 + \gamma^{0}_{j} s_1\) and
\(S_{j} = (1-\gamma^*_{j}) s_0 + \gamma^*_{j} s_1\). To calculate two
unknown quantities \(E({S_j^0}^{-1})\) and \(E(S^{-1}_j)\), the
posterior probability
\(p^0_{j} \equiv \text{Pr}(\gamma^{0}_{j}=1|\Theta^{(t-1)})\) and
\(p_{j} \equiv \text{Pr}(\gamma^*_{j}=1|\Theta^{(t-1)})\) are necessary,
which can be derived via Bayes' theorem. The calculation of \(p_j\) is
slightly different from that of \(p^0_j\), as \(p_j\) depends on the
value of \(\boldsymbol{\beta}_{j}\) and \(p^0_j\) only depends on
\(\beta_j^0\). The calculation follows the equations below.

\[
\begin{aligned}
p_{j}^0 &= \frac{\text{Pr}(\gamma_{j}^0 = 1|\theta_j)f(\beta_{j}^0|\gamma_{j}^0=1, s_1) }{\text{Pr}(\gamma_{j}^0 = 1|\theta_j)f(\beta_{j}^0|\gamma_{j}^0=1, s_1) + \text{Pr}(\gamma_{j}^0 = 0|\theta_j)f(\beta^0_{j}|\gamma^0_{j}=0, s_0)}\\
p_{j} &= \frac{\text{Pr}(\gamma^{*}_{j} = 1|\theta_j)\prod\limits_{k=1}^{K_j}f(\beta_{jk}|\gamma^{*}_{j}=1, s_1) }{\text{Pr}(\gamma^{*}_{j} = 1|\theta_j)\prod\limits_{k=1}^{K_j}f(\beta_{jk}|\gamma^{*}_{j}=1, s_1) + \text{Pr}(\gamma^{*}_{j} = 0|\theta_j)\prod\limits_{k=1}^{K_j}f(\beta_{jk}|\gamma^{*}_{j}=0, s_0)}
\end{aligned}
\]

and
\(\text{Pr}(\gamma_{j}^{0} = 1|\theta_j) = \text{Pr}(\gamma_{j}^*= 1|\theta_j) = \theta_j\),
\(\text{Pr}(\gamma_{j}^{0} = 0|\theta_j) = \text{Pr}(\gamma_{j}^*= 0|\theta_j) = 1-\theta_j\),
\(f(\beta|\gamma=1, s_1) = \text{DE}(\beta|0 , s_1)\),
\(f(\beta|\gamma=0, s_0) = \text{DE}(\beta|0 , s_0)\). It is trivial to
show \begin{align}\label{eq:exp_scale}
&E(\gamma^0_{j})  = p^0_{j} & &E(\gamma^{*}_{j}) = p_{j}\nonumber\\
&E({S^0}^{-1}_{j}) = \frac{1-p^{0}_{j}}{s_0} + \frac{p^{0}_{j}}{s_1} & &E(S^{-1}_{j}) = \frac{1-p_{j}}{s_0} + \frac{p_{j}}{s_1}.
\end{align}

As the formulation of \(E(Q_1)\) can be seen as a \(l_1\) penalized
likelihood function, it can be optimized via coordinate descent
algorithm. After the \(\boldsymbol{\beta}\) are updated with the
coordinate descent algorithm, we update the hypor parameter
\(\boldsymbol{\theta}\). The remaining parameters of interest
\(\boldsymbol{\theta}\) can be updated by maximizing \(E(Q_2)\). As the
beta distribution is a conjugate prior for Bernoulli distribution,
\(\boldsymbol{\theta}\) can be easily updated with a closed form
equation:

\begin{equation}\label{eq:update_theta}
\theta_j = \frac{p^0_j + p_{j} + a - 1 }{a + b}.
\end{equation}

Totally, the framework of the proposed EM-coordinate descent algorithm
was summarized as follows:

\begin{enumerate}
\def\labelenumi{\arabic{enumi})}
\item
  Choose a starting value \(\boldsymbol{\beta}^{(0)}\) and
  \(\boldsymbol{\theta}^{(0)}\) for \(\boldsymbol{\beta}\) and
  \(\boldsymbol{\theta}\). For example, we can initialize
  \(\boldsymbol{\beta}^{(0)} = \boldsymbol{0}\) and
  \(\boldsymbol{\theta}^{(0)} = \boldsymbol{0}.5\)
\item
  Iterate over the E-step and M-step until convergence
\end{enumerate}

E-step: calculate \(E(\gamma^0_{j})\), \(E(\gamma^{*}_{j})\)and
\(E({S^0}^{-1}_{j})\) , \(E({S}^{-1}_{j})\)with estimates of
\(\Theta^{(t-1)}\) from previous iteration

M-step:

\begin{enumerate}
\def\labelenumi{\alph{enumi})}
\item
  Update \(\boldsymbol{\beta}^{(t)}\) and the dispersion parameter
  \(\phi\), if exists, using the coordinate descent algorithm with the
  penalized likelihood function in Equation (\ref{eq:Q1_CD})
\item
  Update \(\boldsymbol{\theta}^{(t)}\) using Equation
  (\ref{eq:update_theta})
\end{enumerate}

We assess convergence by the criterion:
\(|d^{(t)}-d^{(t-1)}|/(0.1+|d^{(t)}|)<\epsilon\), where
\(d^{(t)} = -2\log f(\textbf{y}| \textbf{X}, \boldsymbol{\beta}^{(t)},\phi^{(t)})\)
is the estimate of deviance at the \(t\)th iteration, and \(\epsilon\)
is a small value (say \(10^{-5}\)).

\hypertarget{em-iwls}{%
\subsubsection{EM-IWLS}\label{em-iwls}}

Similar to the EM-CD algorithm, the EM-IWLS algorithm is an iterative
EM-based algorithm where the iterative weighted least squares algorithm
is used to find the estimate of \(\boldsymbol{\beta}, \phi\) that
maximizes \(E(Q_1)\). The iterative weighted least squares algorithm was
originally proposed to fit the classical generalized linear models, and
generalized to fit some Bayesian hierarchical models.\citep{Gelman2013}
Yi and Ma \citep{Yi2012} formulated Student's t-distribution and double
exponential distribution as hierarchical normal distributions such that
generalized linear models with shrinkage priors can be easily fitted
using IWLS in combination with EM algorithm. In this work, we adapt the
EM-IWLS paradigm to fit BHAM with spike-and-slab spline prior, with the
full potential to adopt Student's t, double exponential and mixture
Student's t priors when sparse assumption is not necessary.

A double exponential prior, \(\beta|S \sim DE(0, S)\) can be formulated
as a hierarchical normal prior with unknown variance
\(\tau^2 \sim Gamma(1, 1/(2S^2))\) integrated out: \begin{align*}
  \beta &\sim N(0, \tau^2)\\
  \tau^2 & \sim Gamma(1, 1/(2S^2)), 
\end{align*} For the mixture double exponential priors, we can define
the scale parameter \(S = (1-\gamma)s_0 + \gamma s_1\) following
Equation (\ref{eq:ssl}). The change in the prior formulation in turns
leads to the change in the log posterior density function, as \(Q_1\)
needs to account for the hyper prior of \(\tau^2\):
\begin{equation}\label{eq:Q1_IWLS}
Q_1(\boldsymbol{\beta}, \phi) = \log f(\textbf{y}|\boldsymbol{\beta}, \phi) + \log f(\phi) + \sum\limits_{j=1}^p\left[\log f(\beta^0_j|{\tau^0}^2_j) + \log f({\tau^0}^2_j| S_j^0)+\sum\limits_{k=1}^{K_j} \{\log f(\beta^{*}_{jk}|{\tau^{*}}^2_{jk})+\log f({\tau^*}^2_{jk}| S^*_j)\}\right].
\end{equation} Since \(\boldsymbol{\tau}^2\) are not of our primary
interest, we treat them as the ``missing'' data in addition to the
latent indicators \(\boldsymbol{\gamma}\), and hence construct the
expectation
\(E_{\boldsymbol{\gamma}, \boldsymbol{\tau}^2|\Theta^{(t-1)}}(Q_1)\) in
the E-step. To note, unlike the same latent indicator \(\gamma^*_j\)
which is shared by the coefficients of the non-linear terms
\(\beta^*_{jk}\) for \(k = 1, \dots, K_j\) , \(\tau^2_{jk}\) is
coefficient specific for \(\beta^*_{jk}\).
\(E({S_j^0}^{-1}|\beta_j^0, s_0, s_1), E(S^{-1}_j|\boldsymbol{\beta}_j^*, s_0, s_1), E({\tau^0}^2_{j}|S_j^0, \beta^0_j) \text{ and } E({\tau^*}^2_{jk}|S_j^*, \beta^*_{jk})\)
needs to be calculated to formulate \(E(Q_1)\). As neither
\(E({S_j^0}^{-1}|\beta_j^0, s_0, s_1)\) nor
\(E(S^{-1}_j|\boldsymbol{\beta}_j^*, s_0, s_1)\) depends on \(\tau^2\)s,
they can be derived using Equation (\ref{eq:exp_scale}) via Bayes'
Theorem. On the other hand, \(\tau^{2}\) following a gamma distribution
is a conjugate prior for the normal variance, and the conditional
posterior density of \(\tau^{-2}\) is an inverse Gaussian distribution.
\(E({\tau^0}^{-2}_{j})\) and \(E({\tau^*}^{-2}_{jk})\) are calculated
using the closed form equation \begin{align*}
 E({\tau^0}^{-2}_{j}|S_j^0, \beta^0_j) ={S_j^0}^{-1}/|\beta^0_j| \qquad E({\tau^*}^{-2}_{jk}|S_j^*, \beta^*_{jk})={S_j^*}^{-1}/|\beta^*_{jk}|,
\end{align*} where \(S_j^0\) and \(S_j^*\) are replaced by the
expectation and \(\beta\)s are replaced with \(\beta^{(t-1)}\). With
simplification (up to constant additive terms), we have
\begin{equation}\label{eq:EQ1_IWLS}
E(Q_1) = \log f(\textbf{y}|\boldsymbol{\beta}, \phi) + \log f(\phi) - \sum\limits_{j=1}^p\left[ \frac{({\beta_j^0}^2)}{2\phi E{\tau_j^0}^{2}} +\sum\limits_{k=1}^{K_j} \frac{{\beta_{jk}^*}^2}{2\phi E({\tau_{jk}^*}^{2})}\right].
\end{equation} \(\frac{\beta^2}{2\phi E(\tau^{2})}\) can be seen as the
kernel of a normal density with mean 0 and variance
\(\phi E(\tau^{2})\).

Meanwhile, following the classical IWLS, we can approximate the
generalized model likelihood at each iteration with a weighted normal
likelihood: \[
  f(\textbf{y}|\boldsymbol{\beta}, \phi) \approx MVN(\textbf{z}|X\beta, \Sigma w_i^{-1}\phi)
\] where the `normal response' \(z_i\) and `weight' \(w_i\) are called
the pseudo-response and pseudo-weight respectively.The pseudo data and
the pseudo-weight are calculated by \[
\begin{aligned}
z_i &= \hat\eta_i - \frac{L^{'}(y_i|\hat\eta_i)}{L^{''}(y_i|\hat\eta_i)}& w_i &= - L^{''}(y_i|\hat\eta_i),
\end{aligned}
\] where \(L^{'}(y_i|\hat\eta_i, \hat \phi)\) and
\(L^{''}(y_i|\hat\eta_i, \hat \phi)\) are the first and second
derivative of the log density,
\(\log f(\textbf{y}_i|\boldsymbol{\beta}, \phi)\). With
\(z\sim N(X\beta, \Sigma\phi)\) and
\(\beta \sim N(0, \phi\Sigma_{\tau^2})\), we can augment the two normal
distributions and derive the estimates for \(\boldsymbol{\beta}\) and
\(\phi\) in each step in closed form equations. To note, an estimate of
the variance-covariance matrix can be derived in the EM-IWLS algorithm
and in turn can be used for statistical inferences.

\hypertarget{selecting-optimal-scale-values}{%
\subsection{Selecting Optimal Scale
Values}\label{selecting-optimal-scale-values}}

\label{sec:tune} Our proposed models, BHAM, require two preset scale
parameters (\(s_0\), \(s_1\)). Hence, we need to find the optimal values
for the scale parameters such that the model reaches its best prediction
performance regarding a criteria of preference. This would be achieved
by constructing a two dimensional grid, consists of different pairs of
(\(s_0\), \(s_1\)) value. However, previous research suggested the value
of slab scale \(s_1\) have less impact on the final model and is
recommended to be set as a generally large value, e.g.~\(s_1 = 1\), that
provides no or weak shrinkage. \citep{Rockova2018} As a result, we focus
on examining different values of spike scale \(s_0\). Instead of the 2-D
grid, We consider a sequence of \(L\) decreasing values
\(\{s_0^l\}: 0 < s_0^1 < s_0^2 < \dots < s_0^L < s_1\). Increasing the
spike scale \(s_0\) tends to include more non-zero coefficients in the
model. An measure of preference calculated with cross-validations,
e.g.~deviance, area under the curve (AUC), mean squared error, can be
used to facilitate the selection of a final model. The procedure is
similar to the lasso implemented in the widely used R package
\texttt{glmnet}, which quickly fits the lasso model over a list of
values of \(\lambda\) covering the entire range, giving a sequence of
models for users to choose from.

\hypertarget{simulation-study}{%
\section{Simulation Study}\label{simulation-study}}

\label{sec:sim}

In this section, we compare the performance of the proposed models to
four alternative models: component selection and smoothing operator
(COSSO) \citep{Zhang2006GAM}, adaptive COSSO \citep{Storlie2011},
generalized additive models with automatic smoothing \citep{Wood2011},
SB-GAM \citep{Bai2021}. COSSO is one of the earliest smoothing spline
models that consider sparsity-smoothness penalty. Adaptive COSSO
improved upon COSSO by using adaptive weight for penalties such that the
penalty of each functional component are different for extra
flexibility. Generalized additive models with automatic smoothing,
hereafter \textit{mgcv}, is one of the most popular models for nonlinear
effect interpolation and prediction. SB-GAM is the first spike-and-slab
lasso GAM. We implemented COSSO and adaptive COSSO with R package
\texttt{cosso 2.1-1}, generalized additive models with automatic
smoothing with R package \texttt{mgcv 1.8-31}, SB-GAM with R package
\texttt{sparseGAM 1.0}. COSSO models and SB-GAM do not provide
flexibility to define smoothing functions, and hence used the default
choices. Both mgcv and proposed models allow customized smoothing
functions and we chose the cubic regression spline. We controlled the
dimensionality of each smoothing function, 10 bases, for all different
choices of smoothing functions. 5-fold cross-validation were used for
COSSO models, SB-GAM and the proposed models for tuning parameter
selection based on the default selection criteria. 20 default candidates
of tuning parameters (\(s_0\) in BHAM, \(\lambda_0\) in SB-GAM) were
examined for SB-GAM and the proposed models that allow
user-specification of tuning candidates. All computation was conducted
on a high-performance 64-bit Linux platform with 48 cores of 2.70GHz
eight-core Intel Xeon E5-2680 processors and 24G of RAM per core and R
3.6.2 \citep{R}.

Other related methods for high-dimensional GAMs also exist, notably the
methods of sparse additive models by Ravikumar et al.
\citep{Ravikumar2009}, stochastic search term selection for GAM
\citep{Scheipl2012}. However, we exclude these methods from current
simulation study because of demonstrated inferior predictive performance
compared to mgcv and scalability issues with increased number of
predictors. \citep{Scheipl2013}

\hypertarget{monte-carlo-simulation-study}{%
\subsection{Monte Carlo Simulation
Study}\label{monte-carlo-simulation-study}}

We follow the data generating process described in Bai \citep{Bai2021}.
We first generated \(n=500\) training data points with
\(p=4, 10, 50, 200\) predictors respectively, where the predictors \(X\)
are simulated from a multivariate normal distribution
\(\text{MVN}_{n\times p}(0, I_{P})\). We then simulated the outcome
\(y\) from two distributions, Gaussian and binomial with the identity
link and logit link \(g(x) = \log(\frac{x}{1-x})\) respectively. The
mean of each outcome were simulated via the following function \[
\mathbb{E}(Y) = g^{-1}(5 \sin(2\pi x_1) - 4 \cos(2\pi x_2 -0.5) + 6(x_3-0.5) - 5(x_4^2 -0.3))
\] for gaussian and binomial outcomes. Gaussian outcomes required
specification of dispersion, where we set the dispersion parameter to be
1. In this data generating process, we have \(x_1, x_2, x_3, x_4\) as
the active covariates, while the rest covariates are inactive,
i.e.~\(f_j(x_j) = 0\) for \(j = 4, \dots, p\). Another set of
independent sample of size \(n_{test}=1000\),(\(x_{new} ,y_{new}\)), are
created following the same data generating process, serving as the
testing data. We generated 50 independent pairs of training and testing
datasets to evaluate the prediction performance of the chosen models,
where training datasets are used to fit the models and testing datasets
used to calculate assessment measures.

\begin{table}[ht]
\centering
\begin{tabular}{ccccccc}
  \hline
P & BHAM-IWLS & BHAM-CD & COSSO & Adaptive COSSO & mgcv & SB-GAM \\ 
  \hline
  4 & 0.90 (0.01) & 0.90 (0.01) & 0.73 (0.09) & 0.71 (0.13) & 0.90 (0.01) & 0.82 (0.04) \\ 
   10 & 0.90 (0.01) & 0.88 (0.01) & 0.67 (0.16) & 0.76 (0.04) & 0.90 (0.01) & 0.82 (0.04) \\ 
   50 & 0.88 (0.01) & 0.81 (0.05) & 0.43 (0.19) & 0.54 (0.19) & 0.86 (0.02) & 0.81 (0.07) \\ 
  200 & 0.73 (0.10) & 0.60 (0.08) & 0.39 (0.17) & 0.45 (0.19) & - & 0.81 (0.05) \\ 
   \hline
\end{tabular}
\caption{The average and standard deviation of the out-of-sample $R^2$ measure for Gaussian outcomes over 50 iterations} 
\label{tab:gaus}
\end{table}

\begin{table}[ht]
\centering
\begin{tabular}{rllllll}
  \hline
P & BHAM-IWLS & BHAM-CD & COSSO & Adaptive COSSO & mgcv & SB-GAM \\ 
  \hline
  4 & 0.94 (0.01) & 0.89 (0.04) & 0.90 (0.02) & 0.90 (0.02) & 0.94 (0.01) & 0.93 (0.01) \\ 
   10 & 0.93 (0.01) & 0.87 (0.03) & 0.87 (0.03) & 0.85 (0.03) & 0.92 (0.04) & 0.92 (0.01) \\ 
   50 & 0.92 (0.01) & 0.87 (0.02) & 0.83 (0.02) & 0.83 (0.02) & 0.76 (0.04) & 0.92 (0.01) \\ 
  200 & 0.88 (0.01) & 0.86 (0.02) & 0.81 (0.06) & 0.81 (0.08) & - & 0.92 (0.01) \\ 
   \hline
\end{tabular}
\caption{The average and standard deviation of the out-of-sample area under the curve measure for binomial outcomes over 50 iterations} 
\label{tab:bin_auc}
\end{table}

To evaluate the predictive performance of the models, the statistics
caluclated based on the testing dataset, \(R^2\) for Gaussian model,
area under the curve (AUC) for binomial model, are calculated averaging
across 50 simulations. The results are tabulated in Table
\ref{tab:gaus}, \ref{tab:bin_auc}.

The predictive performances have a consistent pattern across the three
distributions of outcomes. Across all the scenarios, COSSO and adaptive
COSSO have the least favorable performance among the applicable methods
examined. To note, mgcv doesn't support high-dimensional analysis when
the number of coefficients are greater than the sample size. mgcv
predicts well when \(p\) is small or moderate (p = 4, 10), and
deteriorate when the number of predictors increase. Among the three
fast-computing Bayesian hierarchical models, the proposed models,
BHAM-IWLS and BHAM-CD predicts better than SB-GAM when the dimension of
are small, moderate, and high. However, in the hyper high dimensional
case, SB-GAM has better performance. Among the proposed methods,
BHAM-IWLS performs consistently better than BHAM-CD.

~

\hypertarget{metabolomics-data-analysis}{%
\section{Metabolomics Data Analysis}\label{metabolomics-data-analysis}}

\label{sec:real_data}

In this section, we apply the proposed models BHAM-SSL, fitted with the
EM-CD algorithm, to analyze two real-world metabolomics datasets where
the outcomes are binary and continuous respectively. We demonstrate the
improved prediction performance compared to the other Bayesian
hierarchical additive model, SB-GAM \citep{Bai2021}.

\hypertarget{emory-cardiovarcular-biobank}{%
\subsection{Emory Cardiovarcular
Biobank}\label{emory-cardiovarcular-biobank}}

We use the proposed models BHAM-SSL to analyze a metabolic dataset from
a recently published research \citep{Mehta2020} studying plasma
metabolomic profile on the three-year all-cause mortality among patients
undergoing cardiac catheterization. The dataset is publicly available
via \textit{Dryad} \citep{Mehta2020_data}. It contains in total of 776
subjects from two cohorts. As there is a large number of non-overlapping
features among the two cohorts, we use the cohort with larger sample
size (N=454). There were initially 6796 features in the dataset, which
is too large to be practically meaningful to analyze. Hence, we
performed a univariate screening procedure on the features, via GAM
implemented in \texttt{mgcv}, and chose the the top 200 features with
smallest p-values. 5-Knot spline additive models for binary outcome are
constructed using two different models, the proposed BHAM-SSL and the
SB-GAM. 10-Fold cross-validation(CV) are used to choose the optimal
tuning parameters of each framework with respect to the default
selection criterion implemented in the software. Out-of-bag samples were
used for prediction performance evaluation, where deviance, area under
the ROC curve (AUC), mean squared error (MSE) defined as
\(\frac{1}{n}\sum\limits^{n}_{i=1}(y_i - \hat y_i)^2\), and
misclassication error defined as
\(\frac{1}{n}\sum\limits^{n}_{i=1}I(|y_i - \hat y_i|>0.5)\) were
calculated and tabulated in Table \ref{tab:ECB_res}.

\providecommand{\docline}[3]{\noalign{\global\setlength{\arrayrulewidth}{#1}}\arrayrulecolor[HTML]{#2}\cline{#3}}

\setlength{\tabcolsep}{2pt}

\renewcommand*{\arraystretch}{1.5}

\begin{longtable}[c]{|p{1.07in}|p{0.96in}|p{0.67in}|p{0.67in}|p{0.67in}|p{1.39in}}

\caption{Measures of optimal BHAM-SSL and SB-GAM models for Emory Cardiovascular Biobank by 10-fold cross-validation\label{tab:ECB_res}}\\

\hhline{>{\arrayrulecolor[HTML]{666666}\global\arrayrulewidth=2pt}->{\arrayrulecolor[HTML]{666666}\global\arrayrulewidth=2pt}->{\arrayrulecolor[HTML]{666666}\global\arrayrulewidth=2pt}->{\arrayrulecolor[HTML]{666666}\global\arrayrulewidth=2pt}->{\arrayrulecolor[HTML]{666666}\global\arrayrulewidth=2pt}->{\arrayrulecolor[HTML]{666666}\global\arrayrulewidth=2pt}-}

\multicolumn{1}{!{\color[HTML]{000000}\vrule width 0pt}>{\raggedright}p{\dimexpr 1.07in+0\tabcolsep+0\arrayrulewidth}}{\fontsize{11}{11}\selectfont{\textcolor[HTML]{000000}{Methods}}} & \multicolumn{1}{!{\color[HTML]{000000}\vrule width 0pt}>{\raggedleft}p{\dimexpr 0.96in+0\tabcolsep+0\arrayrulewidth}}{\fontsize{11}{11}\selectfont{\textcolor[HTML]{000000}{deviance}}} & \multicolumn{1}{!{\color[HTML]{000000}\vrule width 0pt}>{\raggedleft}p{\dimexpr 0.67in+0\tabcolsep+0\arrayrulewidth}}{\fontsize{11}{11}\selectfont{\textcolor[HTML]{000000}{auc}}} & \multicolumn{1}{!{\color[HTML]{000000}\vrule width 0pt}>{\raggedleft}p{\dimexpr 0.67in+0\tabcolsep+0\arrayrulewidth}}{\fontsize{11}{11}\selectfont{\textcolor[HTML]{000000}{mse}}} & \multicolumn{1}{!{\color[HTML]{000000}\vrule width 0pt}>{\raggedleft}p{\dimexpr 0.67in+0\tabcolsep+0\arrayrulewidth}}{\fontsize{11}{11}\selectfont{\textcolor[HTML]{000000}{mae}}} & \multicolumn{1}{!{\color[HTML]{000000}\vrule width 0pt}>{\raggedleft}p{\dimexpr 1.39in+0\tabcolsep+0\arrayrulewidth}!{\color[HTML]{000000}\vrule width 0pt}}{\fontsize{11}{11}\selectfont{\textcolor[HTML]{000000}{misclassification}}} \\

\noalign{\global\setlength{\arrayrulewidth}{2pt}}\arrayrulecolor[HTML]{666666}\cline{1-6}

\endfirsthead

\hhline{>{\arrayrulecolor[HTML]{666666}\global\arrayrulewidth=2pt}->{\arrayrulecolor[HTML]{666666}\global\arrayrulewidth=2pt}->{\arrayrulecolor[HTML]{666666}\global\arrayrulewidth=2pt}->{\arrayrulecolor[HTML]{666666}\global\arrayrulewidth=2pt}->{\arrayrulecolor[HTML]{666666}\global\arrayrulewidth=2pt}->{\arrayrulecolor[HTML]{666666}\global\arrayrulewidth=2pt}-}

\multicolumn{1}{!{\color[HTML]{000000}\vrule width 0pt}>{\raggedright}p{\dimexpr 1.07in+0\tabcolsep+0\arrayrulewidth}}{\fontsize{11}{11}\selectfont{\textcolor[HTML]{000000}{Methods}}} & \multicolumn{1}{!{\color[HTML]{000000}\vrule width 0pt}>{\raggedleft}p{\dimexpr 0.96in+0\tabcolsep+0\arrayrulewidth}}{\fontsize{11}{11}\selectfont{\textcolor[HTML]{000000}{deviance}}} & \multicolumn{1}{!{\color[HTML]{000000}\vrule width 0pt}>{\raggedleft}p{\dimexpr 0.67in+0\tabcolsep+0\arrayrulewidth}}{\fontsize{11}{11}\selectfont{\textcolor[HTML]{000000}{auc}}} & \multicolumn{1}{!{\color[HTML]{000000}\vrule width 0pt}>{\raggedleft}p{\dimexpr 0.67in+0\tabcolsep+0\arrayrulewidth}}{\fontsize{11}{11}\selectfont{\textcolor[HTML]{000000}{mse}}} & \multicolumn{1}{!{\color[HTML]{000000}\vrule width 0pt}>{\raggedleft}p{\dimexpr 0.67in+0\tabcolsep+0\arrayrulewidth}}{\fontsize{11}{11}\selectfont{\textcolor[HTML]{000000}{mae}}} & \multicolumn{1}{!{\color[HTML]{000000}\vrule width 0pt}>{\raggedleft}p{\dimexpr 1.39in+0\tabcolsep+0\arrayrulewidth}!{\color[HTML]{000000}\vrule width 0pt}}{\fontsize{11}{11}\selectfont{\textcolor[HTML]{000000}{misclassification}}} \\

\noalign{\global\setlength{\arrayrulewidth}{2pt}}\arrayrulecolor[HTML]{666666}\cline{1-6}\endhead



\multicolumn{1}{!{\color[HTML]{000000}\vrule width 0pt}>{\raggedright}p{\dimexpr 1.07in+0\tabcolsep+0\arrayrulewidth}}{\fontsize{11}{11}\selectfont{\textcolor[HTML]{000000}{BHAM-SSL}}} & \multicolumn{1}{!{\color[HTML]{000000}\vrule width 0pt}>{\raggedleft}p{\dimexpr 0.96in+0\tabcolsep+0\arrayrulewidth}}{\fontsize{11}{11}\selectfont{\textcolor[HTML]{000000}{455.686}}} & \multicolumn{1}{!{\color[HTML]{000000}\vrule width 0pt}>{\raggedleft}p{\dimexpr 0.67in+0\tabcolsep+0\arrayrulewidth}}{\fontsize{11}{11}\selectfont{\textcolor[HTML]{000000}{0.740}}} & \multicolumn{1}{!{\color[HTML]{000000}\vrule width 0pt}>{\raggedleft}p{\dimexpr 0.67in+0\tabcolsep+0\arrayrulewidth}}{\fontsize{11}{11}\selectfont{\textcolor[HTML]{000000}{0.159}}} & \multicolumn{1}{!{\color[HTML]{000000}\vrule width 0pt}>{\raggedleft}p{\dimexpr 0.67in+0\tabcolsep+0\arrayrulewidth}}{\fontsize{11}{11}\selectfont{\textcolor[HTML]{000000}{0.285}}} & \multicolumn{1}{!{\color[HTML]{000000}\vrule width 0pt}>{\raggedleft}p{\dimexpr 1.39in+0\tabcolsep+0\arrayrulewidth}!{\color[HTML]{000000}\vrule width 0pt}}{\fontsize{11}{11}\selectfont{\textcolor[HTML]{000000}{0.214}}} \\





\multicolumn{1}{!{\color[HTML]{000000}\vrule width 0pt}>{\raggedright}p{\dimexpr 1.07in+0\tabcolsep+0\arrayrulewidth}}{\fontsize{11}{11}\selectfont{\textcolor[HTML]{000000}{SB-GAM}}} & \multicolumn{1}{!{\color[HTML]{000000}\vrule width 0pt}>{\raggedleft}p{\dimexpr 0.96in+0\tabcolsep+0\arrayrulewidth}}{\fontsize{11}{11}\selectfont{\textcolor[HTML]{000000}{1,230.063}}} & \multicolumn{1}{!{\color[HTML]{000000}\vrule width 0pt}>{\raggedleft}p{\dimexpr 0.67in+0\tabcolsep+0\arrayrulewidth}}{\fontsize{11}{11}\selectfont{\textcolor[HTML]{000000}{0.706}}} & \multicolumn{1}{!{\color[HTML]{000000}\vrule width 0pt}>{\raggedleft}p{\dimexpr 0.67in+0\tabcolsep+0\arrayrulewidth}}{\fontsize{11}{11}\selectfont{\textcolor[HTML]{000000}{0.208}}} & \multicolumn{1}{!{\color[HTML]{000000}\vrule width 0pt}>{\raggedleft}p{\dimexpr 0.67in+0\tabcolsep+0\arrayrulewidth}}{\fontsize{11}{11}\selectfont{\textcolor[HTML]{000000}{0.257}}} & \multicolumn{1}{!{\color[HTML]{000000}\vrule width 0pt}>{\raggedleft}p{\dimexpr 1.39in+0\tabcolsep+0\arrayrulewidth}!{\color[HTML]{000000}\vrule width 0pt}}{\fontsize{11}{11}\selectfont{\textcolor[HTML]{000000}{0.240}}} \\

\noalign{\global\setlength{\arrayrulewidth}{2pt}}\arrayrulecolor[HTML]{666666}\cline{1-6}



\end{longtable}

\hypertarget{weight-loss-maintenance-cohort}{%
\subsection{Weight Loss Maintenance
Cohort}\label{weight-loss-maintenance-cohort}}

We use the proposed models BHAM-SSL to analyze metabolomics data from a
recently published study \citep{Bihlmeyer2021} on the association
between metabolic biomarkers and weight loss, where the dataset is
publically available \citep{Bihlmeyer2021_data}. In this analysis, we
primarily focused on the analysis of one of the three studies included,
weight loss maintenance (WLM) cohort \citep{Svetkey2008}, due to the
drastically different intervention effects. In the dataset, 765
metaboliltes in baseline plasma collected were profiled using liquid
chromatography mass spectrometry. Quality control and natural log
transformation were performed during metabolites data preparation. The
outcome of interest was standardized percent change in insulin
resistance, and hence modeled using a Gaussian model. After removing
missing datapoints and addressing outliers in the data, there were 237
remaining in the analysis. 5-Knot spline additive models for the
gaussian outcome are constructed using two different models, the
proposed BHAM-SSL and the SB-GAM. 10-Fold cross-validation(CV) are used
to choose the optimal tuning parameters of each framework with respect
to the default selection criterion implemented in the software.
Out-of-bag samples were used for prediction performance evaluation,
where deviance, mean squared error (MSE) defined as
\(\frac{1}{n}\sum\limits^{n}_{i=1}(y_i - \hat y_i)^2\), and mean
abosolute error (MAE)defined as
\(\frac{1}{n}\sum\limits^{n}_{i=1}|y_i - \hat y_i|\) were calculated and
tabulated in Table \ref{tab:WLM_res}.

\providecommand{\docline}[3]{\noalign{\global\setlength{\arrayrulewidth}{#1}}\arrayrulecolor[HTML]{#2}\cline{#3}}

\setlength{\tabcolsep}{2pt}

\renewcommand*{\arraystretch}{1.5}

\begin{longtable}[c]{|p{1.07in}|p{0.90in}|p{0.67in}|p{0.67in}|p{0.67in}}

\caption{Measures of optimal BHAM-SSL and SB-GAM models for Weight Loss Maintenance Cohort by 10-fold cross-validation\label{tab:WLM_res}}\\

\hhline{>{\arrayrulecolor[HTML]{666666}\global\arrayrulewidth=2pt}->{\arrayrulecolor[HTML]{666666}\global\arrayrulewidth=2pt}->{\arrayrulecolor[HTML]{666666}\global\arrayrulewidth=2pt}->{\arrayrulecolor[HTML]{666666}\global\arrayrulewidth=2pt}->{\arrayrulecolor[HTML]{666666}\global\arrayrulewidth=2pt}-}

\multicolumn{1}{!{\color[HTML]{000000}\vrule width 0pt}>{\raggedright}p{\dimexpr 1.07in+0\tabcolsep+0\arrayrulewidth}}{\fontsize{11}{11}\selectfont{\textcolor[HTML]{000000}{Methods}}} & \multicolumn{1}{!{\color[HTML]{000000}\vrule width 0pt}>{\raggedleft}p{\dimexpr 0.9in+0\tabcolsep+0\arrayrulewidth}}{\fontsize{11}{11}\selectfont{\textcolor[HTML]{000000}{deviance}}} & \multicolumn{1}{!{\color[HTML]{000000}\vrule width 0pt}>{\raggedleft}p{\dimexpr 0.67in+0\tabcolsep+0\arrayrulewidth}}{\fontsize{11}{11}\selectfont{\textcolor[HTML]{000000}{R2}}} & \multicolumn{1}{!{\color[HTML]{000000}\vrule width 0pt}>{\raggedleft}p{\dimexpr 0.67in+0\tabcolsep+0\arrayrulewidth}}{\fontsize{11}{11}\selectfont{\textcolor[HTML]{000000}{mse}}} & \multicolumn{1}{!{\color[HTML]{000000}\vrule width 0pt}>{\raggedleft}p{\dimexpr 0.67in+0\tabcolsep+0\arrayrulewidth}!{\color[HTML]{000000}\vrule width 0pt}}{\fontsize{11}{11}\selectfont{\textcolor[HTML]{000000}{mae}}} \\

\noalign{\global\setlength{\arrayrulewidth}{2pt}}\arrayrulecolor[HTML]{666666}\cline{1-5}

\endfirsthead

\hhline{>{\arrayrulecolor[HTML]{666666}\global\arrayrulewidth=2pt}->{\arrayrulecolor[HTML]{666666}\global\arrayrulewidth=2pt}->{\arrayrulecolor[HTML]{666666}\global\arrayrulewidth=2pt}->{\arrayrulecolor[HTML]{666666}\global\arrayrulewidth=2pt}->{\arrayrulecolor[HTML]{666666}\global\arrayrulewidth=2pt}-}

\multicolumn{1}{!{\color[HTML]{000000}\vrule width 0pt}>{\raggedright}p{\dimexpr 1.07in+0\tabcolsep+0\arrayrulewidth}}{\fontsize{11}{11}\selectfont{\textcolor[HTML]{000000}{Methods}}} & \multicolumn{1}{!{\color[HTML]{000000}\vrule width 0pt}>{\raggedleft}p{\dimexpr 0.9in+0\tabcolsep+0\arrayrulewidth}}{\fontsize{11}{11}\selectfont{\textcolor[HTML]{000000}{deviance}}} & \multicolumn{1}{!{\color[HTML]{000000}\vrule width 0pt}>{\raggedleft}p{\dimexpr 0.67in+0\tabcolsep+0\arrayrulewidth}}{\fontsize{11}{11}\selectfont{\textcolor[HTML]{000000}{R2}}} & \multicolumn{1}{!{\color[HTML]{000000}\vrule width 0pt}>{\raggedleft}p{\dimexpr 0.67in+0\tabcolsep+0\arrayrulewidth}}{\fontsize{11}{11}\selectfont{\textcolor[HTML]{000000}{mse}}} & \multicolumn{1}{!{\color[HTML]{000000}\vrule width 0pt}>{\raggedleft}p{\dimexpr 0.67in+0\tabcolsep+0\arrayrulewidth}!{\color[HTML]{000000}\vrule width 0pt}}{\fontsize{11}{11}\selectfont{\textcolor[HTML]{000000}{mae}}} \\

\noalign{\global\setlength{\arrayrulewidth}{2pt}}\arrayrulecolor[HTML]{666666}\cline{1-5}\endhead



\multicolumn{1}{!{\color[HTML]{000000}\vrule width 0pt}>{\raggedright}p{\dimexpr 1.07in+0\tabcolsep+0\arrayrulewidth}}{\fontsize{11}{11}\selectfont{\textcolor[HTML]{000000}{BHAM-SSL}}} & \multicolumn{1}{!{\color[HTML]{000000}\vrule width 0pt}>{\raggedleft}p{\dimexpr 0.9in+0\tabcolsep+0\arrayrulewidth}}{\fontsize{11}{11}\selectfont{\textcolor[HTML]{000000}{669.375}}} & \multicolumn{1}{!{\color[HTML]{000000}\vrule width 0pt}>{\raggedleft}p{\dimexpr 0.67in+0\tabcolsep+0\arrayrulewidth}}{\fontsize{11}{11}\selectfont{\textcolor[HTML]{000000}{0.077}}} & \multicolumn{1}{!{\color[HTML]{000000}\vrule width 0pt}>{\raggedleft}p{\dimexpr 0.67in+0\tabcolsep+0\arrayrulewidth}}{\fontsize{11}{11}\selectfont{\textcolor[HTML]{000000}{0.924}}} & \multicolumn{1}{!{\color[HTML]{000000}\vrule width 0pt}>{\raggedleft}p{\dimexpr 0.67in+0\tabcolsep+0\arrayrulewidth}!{\color[HTML]{000000}\vrule width 0pt}}{\fontsize{11}{11}\selectfont{\textcolor[HTML]{000000}{0.745}}} \\





\multicolumn{1}{!{\color[HTML]{000000}\vrule width 0pt}>{\raggedright}p{\dimexpr 1.07in+0\tabcolsep+0\arrayrulewidth}}{\fontsize{11}{11}\selectfont{\textcolor[HTML]{000000}{SB-GAM}}} & \multicolumn{1}{!{\color[HTML]{000000}\vrule width 0pt}>{\raggedleft}p{\dimexpr 0.9in+0\tabcolsep+0\arrayrulewidth}}{\fontsize{11}{11}\selectfont{\textcolor[HTML]{000000}{666.826}}} & \multicolumn{1}{!{\color[HTML]{000000}\vrule width 0pt}>{\raggedleft}p{\dimexpr 0.67in+0\tabcolsep+0\arrayrulewidth}}{\fontsize{11}{11}\selectfont{\textcolor[HTML]{000000}{0.026}}} & \multicolumn{1}{!{\color[HTML]{000000}\vrule width 0pt}>{\raggedleft}p{\dimexpr 0.67in+0\tabcolsep+0\arrayrulewidth}}{\fontsize{11}{11}\selectfont{\textcolor[HTML]{000000}{0.976}}} & \multicolumn{1}{!{\color[HTML]{000000}\vrule width 0pt}>{\raggedleft}p{\dimexpr 0.67in+0\tabcolsep+0\arrayrulewidth}!{\color[HTML]{000000}\vrule width 0pt}}{\fontsize{11}{11}\selectfont{\textcolor[HTML]{000000}{0.766}}} \\

\noalign{\global\setlength{\arrayrulewidth}{2pt}}\arrayrulecolor[HTML]{666666}\cline{1-5}



\end{longtable}

\hypertarget{discussion}{%
\section{Discussion}\label{discussion}}

\label{sec:concl} In the paper, we described a novel generalized
additive model using Bayesian hierarchical priors, particularly the
proposed spike-and-slab spline prior for bi-level functional selection.
Meanwhile, we introduced two optimization based algorithms for model
fitting. The algorithms can be easily scale up to address
high-dimensional data analysis in a computational efficient manner, but
also provide naive inferential supports for many commonly used
hierarchical priors, e.g.~t-distribution, double exponential
distribution, mixture t distribution, and mixture double exponential
distribution, in the analyses that the sparse assumption is weak or not
necessary. Via simulations, we demonstrated that the proposed model
provides as good, if not better, prediction performance compared to some
state-of-the-art non-linear modeling devices.

The proposed model shares many commonality with an high-dimensional
Bayesian GAM, SB-GAM \citep{Bai2021}, independently developed around the
same time of this work. Both frameworks emphasize computational
efficiency by deploying group spike-and-slab lasso type of priors and
optimization-based fast and scalable algorithms. Bai provided the
theoretical proof for the consistency of variable selection using group
spike-and-slab lasso prior. Nevertheless, SB-GAM fails to address the
bi-level functional selection and the model inference. The proposed
model provides solutions to these remaining questions while maintaining
the same level of prediction accuracy. Moreover, the proposed model
renders additional flexibility to model specification, allowing various
choices of smoothing functions and degrees of freedom.

For transnational science purpose, we implemented the proposed model in
a R package \texttt{BHAM}, deposited at
\url{https://github.com/boyiguo1/BHAM}. To maximize the flexibility of
smoothing function specification, we deploy the same programming grammar
as in the state-of-the-art package \texttt{mgcv}, in contrast to
previous tools where smoothing functions are limited to the default
ones. Ancillary functions are provided for model specification in
high-dimensional settings, curve plotting and functional selection.

There are some improvements possible for the proposed models. First of
all, the proposed model achieves a bi-level selection via the two-part
spike-and-slab spline prior. Nevertheless, this set-up could result in a
situation that is not theoretically sound: the non-linear component is
selected, but the linear component is not. We currently address it
analytically by forcing to include the linear component when non-linear
component is included. Another possible solution is to impose an
dependent structure of \(\gamma_{j}^*\) on \(\gamma_{j^{0}}\),
i.e.~\(\gamma_j^*|\gamma_{j}^{0}, \theta_j\). Secondly, while the
proposed model addresses a great deal of analytic problem, analyzing the
time-to-event outcome remains unsolved. Our current effort resides on
expanding GAM to Cox proportional hazard models. An naive approach would
be convert an time-to-event outcome to a poisson outcome following
Whitehead \citep{Whitehead1980}. However, it would be more efficient to
directly fitting Cox models via penalized pseudo likelihood function.
Last but not least, we plan to expand the proposed model to integrate
external biology information. The main motivation for the integrative
model is that biologically informed grouping of weak effects increases
the power of detecting true associations between features and the
outcome \citep{Peterson2016}, and stabilizes the analysis results for
reproducibility purpose. Such integration can be achieved by setting up
a structural hyperprior on the inclusion indicator of the smoothing
function null space \(\boldsymbol{\gamma}^0\). The similar strategy has
been used in Ferrari and Dunson \citep{Ferrari2020}.

\bibliography{bibfile.bib}


\end{document}
